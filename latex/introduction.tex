\section{Γενικό Πλαίσιο}
\drop{S}{ε} αυτή την εργασία παρουσιάζεται η διαδικασία ανάπτυξης μίας ψηφιακής πλατφόρμας όπου στοχεύει στην βελτίωση του τρόπου όπου οι δήμοι των πόλεων ανταποκρίνονται και διαχειρίζονται προβλήματα που έχουν να κάνουν με τις υποδομές τους όταν αυτά αναφέρονται από τους πολίτες. Πιο συγκεκριμένα, το έργο εστιάζει στην δημιουργία ενός πρακτικού και αποδοτικού εργαλείου επικοινωνίας μεταξύ των κατοίκων και των τοπικών αρχών μέσω μίας εφαρμογής για κινητά και μίας ιστοσελίδας διαχείρισης. Πριν την εξέταση των τεχνικών διαδικασιών, είναι σημαντική η κατανόηση του ευρύτερου πλαισίου του θέματος, σε ποιο βαθμό ένα τέτοιο σύστημα θεωρείται απαραίτητο καθώς και πόσο βοηθάει στην διαχείριση ενός δήμου και στην συμμετοχή των πολιτών σε αυτόν.
\\
\\
Σε πολλές πόλεις, ειδικά σε μικρότερες, οι τρόποι διαχείρισης όπου χρησιμοποιούνται συνήθως αποτελούνται από ξεπερασμένες ή μη λειτουργικές διαδικασίες. Οι ίδιοι οι κάτοικοι συνήθως αναφέρουν προβλήματα όπως σπασμένα φώτα δρόμου και λακκούβες είτε μέσω από φυσικών επισκέψεων ή τηλεφωνικών κλήσεων. Αυτές οι προσεγγίσεις όχι μόνο κάνουν δύσκολο για τις υπηρεσίες του δήμου να μπορέσουν να διατηρήσουν μία οργάνωση των ενεργών αναφορών αλλά επίσης καθυστερούν την επίλυση αυτών των προβλημάτων \parencite{shama2024citysolution}. Συνεπώς, η έλλειψη ενός ολοκληρωμένου συστήματος οδηγεί σε ανεπαρκή αποτελεσματικότητα και μειωμένη έως και μηδενική επικοινωνία. Από την μεριά των πολιτών, η διαδικασία αυτή είναι χρονοβόρα και δεν ανταποδίδει πραγματικά και ουσιώδη αποτελέσματα, κάτι το οποίο τους προκαλεί απογοήτευση και τους οδηγεί στην αδιαφορία της τελείως.

Την ίδια στιγμή όμως, στην σημερινή εποχή οι άνθρωποι έχουν συνηθίσει την ευκολία του ψηφιακού κόσμου στην καθημερινότητα της ζωής τους. Από την πληρωμή λογαριασμών μέχρι και την πρόσβαση σε υπηρεσίες της κυβέρνησης μέσω του διαδικτύου, έχει γίνει πλέον απαραίτητη η ύπαρξη ενός γρήγορου, εύκολου και φιλικού προς τον χρήστη συστηματος. Οι τοπικές αρχές χρειάζονται να προσαρμόσουνε τις υπηρεσίες τους για να μπορέσουν να καλύψουν αυτές τις προσδοκίες και να προωθήσουν την συμμετοχή της κοινότητας τους παραπάνω. Παρ’ όλα αυτά, η μετάβαση προς την ψηφιακή διαχείριση μέσω εργαλείων συχνά εμποδίζεται λόγω περιορισμένων πόρων και το επίπεδο τεχνικής εμπειρίας. Αυτό οδηγεί σε ένα ψηφιακό κενό μεταξύ τι μπορεί να προσφέρει η τεχνολογία και τι πραγματικά υλοποιούν οι δήμοι.
\\
\\
Μέσω αυτού του έργου, ο σκοπός είναι αυτό το κενό να καλυφθεί. Η ιδέα είναι  η δημιουργία μίας πλατφόρμας όπου αξιοποιεί σύγχρονες τεχνολογίες τόσο του ιστού όσο και των κινητών για να γίνει η επικοινωνία πολίτη και πόλης ευκολότερη και γρηγορότερη. Όπως συνοψίζουν οι \textcite{aslam2020comprehensive}, οι έξυπνες πόλεις συνήθως υλοποιούν κοινή υποδομή δεδομένων που επιτρέπει τον συγχρονισμό των πληροφοριών σε πραγματικό χρόνο. Επομένως, με την εισαγωγή μίας απλής πλατφόρμας αναφορών, η οποία υποστηρίζεται από κοινή βάση δεδομένων, οι άνθρωποι και οι εργάτες του δήμου μπορούν να επωφεληθούν από την βελτιωμένη αποτελεσματικότητα. Από την μία πλευρά, οι κάτοικοι παίρνουν πρόσβαση σε ένα τρόπο ώστε να συνεισφέρουν στην συντήρηση της πόλης τους. Από την άλλη, οι τοπικές αρχές αποκτούν ένα καλά δομημένο εργαλείο για την διαχείριση των δεδομένων τους, το οποίο τους διευκολύνει στην δουλειά τους.

\section{Το πρόβλημα}
Μία από τις πιο σημαντικές προκλήσεις στις τοπικές διοικήσεις στην σημερινή εποχή είναι η διατήρηση της επικοινωνίας τους μαζί με τους πολίτες. Παρά την διάθεση τεράστιου αριθμού ψηφιακών εργαλείων σε πολλούς τομείς, αυτές συνήθως βασίζονται σε ξεπερασμένα συστήματα ή χειροκίνητες διαδικασίες που κάνουν δύσκολο την διαχείριση των αστικών προβλημάτων αποτελεσματικά. Οι κάτοικοι όπου παρατηρούν προβλήματα στις γειτονιές τους συχνά δεν έχουν κάποιο διαθέσιμο και ευθύ μέσο να τα αναφέρουν. Στις πιο πολλές περιπτώσεις, αυτοί πρέπει να επικοινωνήσουν με τον δήμο μέσω κινητού ή φυσικής παρουσίας, όπου και τα δύο είναι ανεπαρκή και χρονοβόρα. Ως αποτέλεσμα, πολλά προβλήματα παραμένουν άφτιαχτα για μεγάλη περίοδο χρόνου, ενώ οι αρχές ζορίζονται στο να κρατήσουν έλεγχο και προτεραιότητα των διάφορων αναφορών όπου λαμβάνουν συνεχώς.
\\
\\
Η απουσία ενός τέτοιου βασικού μηχανισμού οδηγεί σε πολλαπλά θέματα. Κάποια δεδομένα μπορεί να χαθούν, η επικοινωνία γίνεται ασυνεπής και οι κάτοικοι δεν έχουν κάποιο τρόπο να παρακολουθούν την πρόοδο των αιτημάτων τους. Οι μικροί δήμοι συχνά δεν έχουν τους απαραίτητους πόρους ώστε να επενδύσουν σε λύσεις πολύπλοκων λογισμικών ή ειδικών συστημάτων, κάτι το οποίο τους αφήνει εξαρτημένους σε ακατάλληλες μεθόδους εργασίας. Αυτή η κατάσταση, όχι μόνο περιορίζει την αποδοτικότητα των υπηρεσιών τους αλλά και μειώνει την εμπιστοσύνη και την συμμετοχή του κόσμου. Οι κάτοικοι νιώθουν πως δεν είναι μέρος της διαδικασίας και οι αρχές χάνουν την ανατροφοδότηση που χρειάζονται για να βελτιωθούν.
\\
\\
Το ερευνητικό πρόβλημα που αντιμετωπίζεται μέσα σε αυτή την εργασία προκύπτει από αυτήν ακριβώς την αποσύνδεση μεταξύ των δύο. Η ανάγκη δηλαδή για μία ενωμένη, απλή και εύκολα προσβάσιμη πλατφόρμα όπου βασίζεται πάνω στην επικοινωνία του πολίτη με τον δήμο. Ένα τέτοιο σύστημα πρέπει να είναι φτιαγμένο για καθημερινούς χρήστες ενώ προσφέρει στους διαχειριστές εργαλεία για αποδοτική κατηγοριοποίηση και παρακολούθηση των καταχωρήσεων. Η πρόκληση οπότε δεν αφορά μόνο την κατασκευή μίας τεχνολογικής υποδομής αλλά και στην διασφάλιση ότι προωθεί την εμπλοκή των χρηστών. Επομένως, το πρόβλημα είναι τεχνικό και κοινωνικό και αφορά το πως θα χτιστεί ένα σύστημα όπου ενισχύει την επικοινωνία ενώ παραμένει εύκολο και ιδανικό για πόλεις μικρού μεγέθους.
\\
\\
Από μία ερευνητική πλευρά, εξετάζεται πως μπορούν να ενσωματωθούν διαθέσιμα και σύγχρονα εργαλεία όπως το Ionic React για την αντιμετώπιση τέτοιων πραγματικών δημοτικών προκλήσεων. Διερευνώνται οι αρχές σχεδιασμού όπου χρειάζονται για την δημιουργία επεκτάσιμων εφαρμογών με επίκεντρο τον χρήστη και βελτιώνουν την ροή πληροφοριών. Αυτή η έρευνα είναι σημαντική ώστε να δείξει πως η ψηφιακή πρόοδος δεν πρέπει μόνο να ωφελεί τις μεγάλες πόλεις αλλά να μπορεί και να βοηθήσει μικρές κοινότητες να αναπτυχθούν. Άρα, ο κύριος σκοπός είναι η επίδειξη πως ο προσεκτικός σχεδιασμός και η επιλογή κατάλληλων σύγχρονων τεχνολογιών μπορούν να μεταμορφώσουν μία μη αποδοτική διαδικασία σε ένα οργανωμένο και λειτουργικό σύστημα.

\section{Στόχοι και Ερωτήματα}
Ο κύριος στόχος είναι η σχεδίαση και η ανάπτυξη μίας λειτουργικής πλατφόρμας η οποία είναι ικανή να βοηθήσει τους δήμους να ανταποκρίνονται σε αστικά προβλήματα πιο αποτελεσματικά με την συνεργασία των κατοίκων. Η πλατφόρμα συνδυάζει μία κινητή εφαρμογή για τους πολίτες και μία διεπαφή ιστού για τους διαχειριστές, χρησιμοποιώντας μία κοινή βάση δεδομένων. Σκοπός είναι η παροχή στους πολίτες έναν εύκολο τρόπο να μπορούν να αναφέρουν θέματα στην πόλη τους και στις αρχές των δήμων ένα δομημένο τρόπο να λαμβάνουν και να ενεργούν πάνω σε αυτές τις πληροφορίες. Υλοποιώντας αυτό το σύστημα, αποδεικνύεται το πως μοντέρνες τεχνολογίες μπορούν να ενισχύσουν την αποδοτικότητα σε μία κοινότητα.
\\
\\
Από μία πιο γενική εικόνα, η εργασία επικεντρώνεται στην προώθηση της συμμετοχής και της συνεργασίας μέσω της τεχνολογίας. Με την παραχώρηση απλών ψηφιακών εργαλείων, είναι πιο πιθανή η αλληλεπίδραση στην αναφορά θεμάτων καθώς και βελτιώνεται η οργάνωση και ο χρόνος απόκρισης σε αυτά. Επομένως, οι στόχοι δεν παραμένουν μόνο τεχνικοί αλλά και κοινωνικοί.

Για την επίτευξη αυτών των στόχων, ακολουθείται μία ροή προσανατολισμένη στην ανάπτυξη. Πιο συγκεκριμένα:

\begin{itemize}
    \item Η ανάλυση των βασικών προκλήσεων όπου αντιμετωπίζουν οι δήμοι.
    \item Η σχεδίαση μίας πλατφόρμας όπου συνδέει άμεσα τις δύο πλευρές του θέματος.
    \item Η υλοποίηση του συστήματος αυτού χρησιμοποιώντας μοντέρνες τεχνολογίες.
    \item Η δοκιμή όσον αφορά την χρηστικότητα και λειτουργικότητα, διασφαλίζοντας ότι μπορεί πραγματικά να χρησιμοποιηθεί.
    \item Η εκτίμηση του πως ένα τέτοιο σύστημα μπορεί να βοηθήσει σε σχέση με παραδοσιακούς μεθόδους.
\end{itemize}

Επιπρόσθετα, κάποιες ερωτήσεις όπου προκύπτουν με βάση αυτούς τους στόχους είναι:

\begin{itemize}
    \item Ποιες αρχές σχεδιασμού και τεχνικά frameworks είναι πιο κατάλληλα για την δημιουργία μίας τέτοιας πλατφόρμας;

    \item Πως μπορεί το σύστημα να παραμείνει απλό και προσβάσιμο παρέχοντας παράλληλα βασικές λειτουργίες;

    \item Με ποιους τρόπους το αποτέλεσμα προωθεί την συμμετοχικότητα;
\end{itemize}

Απαντώντας αυτές τις ερωτήσεις, μπορεί να δημιουργηθεί ένα πρωτότυπο. Το αποτέλεσμα εκτιμάται να δείξει πως η στοχευμένη χρήση της τεχνολογίας μπορεί να απλοποιήσει την καθημερινότητα και να βελτιώσει την αξιοπιστία σε μία κοινωνία.

\section{Δομή Εργασίας}
Η εργασία έχει οργανωθεί σε έξι κεφάλαια, με το κάθε ένα να εστιάζει σε ένα διαφορετικό μέρος της προτεινόμενης ψηφιακής πλατφόρμας. Η δομή αυτή στοχεύει στην σταδιακή καθοδήγηση κάθε βήματος που έγινε από την κατανόηση του θέματος και του προβλήματος, μέχρι και την τελική υλοποίηση και τα συμπεράσματα. Μετά το εισαγωγικό κομμάτι της, προχωράει στο θεωρητικό υπόβαθρο όπου εξερευνεί τις κύριες έννοιες και τεχνολογίες όπου είναι σχετικές και ταιριάζουν με το έργο. Αυτό το κεφάλαιο παρέχει την απαραίτητη βάση για την κατανόηση του σκεπτικού του συστήματος και της χρήσης του. 
\\
\\
Μετά την θεωρητική συζήτηση, το θέμα αλλάζει στο πιο τεχνικό κομμάτι αρχίζοντας την ανάλυση και την φάση του σχεδιασμού. Σε αυτό το κομμάτι, αναγνωρίζονται οι απαραίτητες απαιτήσεις και χτίζεται μία λειτουργική αρχιτεκτονική. Ακόμα, εξετάζεται το πως δομήθηκε το σύστημα, οι σχέσεις μεταξύ των λειτουργιών του και οι αποφάσεις σχεδίασης όπου διαμορφώνουν την ανάπτυξη του.
\\
\\
Προχωρώντας, παρουσιάζεται η διαδικασία υλοποίησης, δίνοντας μία αναλυτική εξήγηση για το πως κάθε μέρος αναπτύχθηκε και συνδέθηκε ώστε να δημιουργηθεί ένα πλήρως λειτουργικό σύστημα. Ακόμα, περιγράφεται η λογική πίσω από κάθε κώδικα αναλύοντας πως δουλεύει και με ποιο τρόπο γίνεται η μεταφορά των δεδομένων μεταξύ της εφαρμογής στην βάση δεδομένων και την ιστοσελίδα. Γενικά, αυτό το μέρος στοχεύει να δώσει μία τεχνική εικόνα του συστήματος ως ένα σύνολο και να εξηγήσει πως αυτό εκτελείται.
\\
\\
Το τελευταίο κομμάτι επικεντρώνεται πάνω στην αξιολόγηση του τελικού αποτελέσματος και αντανακλά το πόσο κοντά αυτό φτάνει και εκπληρώνει την εκτίμηση και τους στόχους όπου τέθηκαν από την αρχή. Τέλος, αναφέρονται τα συμπεράσματα της όλης διαδικασίας καθώς και προτείνονται πιθανές βελτιώσεις για περαιτέρω ανάπτυξη για μία ακόμη πιο ολοκληρωμένη λύση πάνω στο θέμα.