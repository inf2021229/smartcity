Σε αυτή την πτυχιακή εργασία σχεδιάζεται και υλοποιείται μία πλατφόρμα αναφορών για έξυπνες πόλεις που ενώνει τους κατοίκους και τις υπηρεσίες του δήμου ώστε να διαχειριστούν καθημερινά αστικά προβλήματα πιο εύκολα και αποδοτικά. Σκοπός της είναι να αντικαταστήσει τις αργές και χειροκίνητες μεθόδους, όπως κλήσεων μέσω τηλεφώνου και φυσικών επισκέψεων, με μία απλούστερη ροή. Η συνολική ιδέα είναι ότι οι πολίτες θα παίρνουν μία φωτογραφία του προβλήματος και την υποβάλλουν στο σύστημα. Έπειτα, οι αρχές διαχειρίζονται τις αναφορές αυτές, οι οποίες εμφανίζονται σε έναν διαδραστικό χάρτη ώστε και οι δύο πλευρές να μπορούν να παρακολουθούν την πρόοδο τους. Η πλατφόρμα χρησιμοποιεί ένα backend φτιαγμένο με Node.js και Express, μία cloud βάση δεδομένων μέσω του MongoDB Atlas, το framework Ionic React για την εφαρμογή για τα κινητά και την βιβλιοθήκη Leaflet για την οπτικοποίηση του χάρτη στην ιστοσελίδα. Η εργασία καλύπτει τις θεωρητικές απαιτήσεις, την αρχιτεκτονική και την υλοποίηση ενός τέτοιου συστήματος. Αρχιτεκτονικά, η πλατφόρμα ακολουθεί ένα σχέδιο τριών επιπέδων, διαχωρίζοντας το κάθε τμήμα της ξεχωριστά και κρατώντας τα όλα συνδεδεμένα μέσω ενός δομημένου API για να βοηθήσει στην συντήρηση και την διευκόλυνση της ανάπτυξης. Το τελικό αποτέλεσμα αξιολογήθηκε από 27 συμμετέχοντες χρησιμοποιώντας ένα ερωτηματολόγιο απαντήσεων από ένα έως πέντε της κλίμακας Likert. Τα αποτελέσματα έδειξαν υψηλές βαθμολογίες και μία θετική συνολική εμπειρία, με τους ερωτώντες να δείχνουν πρόθεση πως θα συνέχιζαν να χρησιμοποιούν μία τέτοια πλατφόρμα και ότι αυτή καταφέρνει τον στόχο της να ενώσει τους πολίτες με τους δήμους των πόλεων.
