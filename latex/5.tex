\section{Σκοπός Πειραματισμού}
\drop{T}{ο} κεφάλαιο αυτό έχει να κάνει με την πειραματική διαδικασία που έγινε ώστε να αξιολογηθεί η πλατφόρμα της έξυπνης πόλης όσον αφορά την λειτουργικότητα, την χρηστικότητα και την απόδοση της. Ακολουθώντας την ολοκλήρωση της φάσης της υλοποίησης, μία σειρά από δοκιμές έγιναν ώστε να βεβαιωθεί πως η κινητή εφαρμογή και η ιστοσελίδα λειτουργούν σωστά, επικοινωνούν αποτελεσματικά με το backend και την βάση δεδομένων και υπάρχει μία ομαλή εμπειρία χρήστη.
\\
\\
Ο πειραματισμός εστιάζει πάνω σε τρεις βασικές φάσεις. Η πρώτη φάση έχει να κάνει με την λειτουργική επαλήθευση, δηλαδή την επιβεβαίωση πως όλες οι βασικές λειτουργίες του συστήματος όπως η υποβολή αναφοράς και η απεικόνιση του χάρτη και των καταχωρήσεων σε αυτόν, δουλεύουν όπως είχαν προοριστεί. Το δεύτερο κομμάτι αφορά την αξιολόγηση από εξωτερικούς, από την ανάπτυξη, χρήστες που είδαν την πλατφόρμα και προσφέρει μία καθαρή και ανεξάρτητη γνώμη σχετικά με τον σχεδιασμό και την εμπειρία του συστήματος. Η τρίτη και τελευταία, εστιάζει στην συζήτηση των ευρημάτων από τα αποτελέσματα των χρηστών σε σχέση με την αρχική εκτίμηση.
\\
\\
Μέσω του συνδυασμού των τριών μεθόδων, παρέχονται αντικειμενικές ενδείξεις σωστής λειτουργίας αλλά και εσωτερικές γνώσεις σχετικά με την ποιότητα του συστήματος και την ευχρηστία του. Έτσι, μπορούν να βγουν κατανοητά αποτελέσματα, τα οποία θα συγκριθούν με τους αρχικού στόχους. 

\section{Διαδικασία Πειραματισμού και Αποτελέσματα}

\subsection{Λειτουργική Επιβεβαίωση}
Με σκοπό να επιβεβαιωθεί ότι το σύστημα δουλεύει όπως αναμένεται, ένα σύνολο από στοχευμένα λειτουργικά τεστ πραγματοποιήθηκε πάνω στα βασικά χαρακτηριστικά της κινητής εφαρμογής και της ιστοσελίδας.
\\
\\
Το τελικό τεστ έγινε σε πραγματική Android συσκευή χρησιμοποιώντας την ανεπτυγμένη εφαρμογή. Η ιστοσελίδα ανέβηκε με την χρήση μίας δωρεάν υπηρεσίας φιλοξενίας και χρησιμοποιήθηκε μέσω προγράμματος περιήγησης στον υπολογιστή. Τα δύο αυτά συστήματα επικοινωνούσαν μέσω του backend όπου έτρεχε μαζί με την ιστοσελίδα και εξασφάλιζαν έτσι σύνδεση με την cloud βάση δεδομένων.
\\
\\
Κάθε δοκιμή εστιάζει σε μία κρίσιμη ενέργεια στην ροή εργασίας της πλατφόρμας, από την υποβολή έως και την ανάκτηση των δεδομένων και τις λειτουργίες του διαχειριστή. Όπως φαίνεται στον πίνακα \ref{tab:functional_tests}, υπάρχει ένα αναμενόμενο και ένα πραγματικό αποτέλεσμα για κάθε περίπτωση δοκιμής και αν αυτό μπορεί να θεωρηθεί δεκτό.
\\
\\
Όλα τα λειτουργικά τεστ ολοκληρώθηκαν με επιτυχία, επιβεβαιώνοντας ότι η κύρια επικοινωνία μεταξύ της κινητής εφαρμογής και της ιστοσελίδας, του API και της βάσης δεδομένων δουλεύουν με αξιοπιστία.
\\
\\
Κανένα σημαντικό σφάλμα ή κόλλημα δεν εμφανίστηκε και όλες οι διαδικασίες μεταφοράς δεδομένων εκτελέστηκαν όπως αναμενόταν. Μία μικρή καθυστέρηση παρατηρήθηκε μόνο όταν γίνεται η πρώτη ενέργεια κλήσης του API, κάτι το οποίο είναι συνηθισμένο σε δωρεάν υπηρεσίες φιλοξενίας ιστοτόπων καθώς αν υπάρχει απραξία του backend ή της ιστοσελίδας για αρκετή ώρα, αυτή αυτόματα μπαίνει σε κατάσταση αδράνειας.
\begin{table}[H]
\centering
\caption{Αποτελέσματα ελέγχων λειτουργικότητας της πλατφόρμας}
\label{tab:functional_tests}
\renewcommand{\arraystretch}{1.5}
\setlength{\tabcolsep}{6pt}
\begin{tabular}{|p{4cm}|p{4.5cm}|p{4.5cm}|p{1.7cm}|}
\hline
\textbf{Περίπτωση Ελέγχου} & \textbf{Αναμενόμενο Αποτέλεσμα} & \textbf{Πραγματικό Αποτέλεσμα} & \textbf{Πέτυχε;} \\ \hline
Εγγραφή και σύνδεση χρήστη & 
Ο λογαριασμός δημιουργείται και η ταυτοποίηση ολοκληρώνεται &
Ο λογαριασμός δημιουργήθηκε και η σύνδεση έγινε με επιτυχία &
Ναι \\ \hline
Υποβολή αναφοράς (φωτογραφία, περιγραφή, GPS) &
Η αναφορά αποθηκεύεται στη βάση δεδομένων &
Η αναφορά αποθηκεύτηκε και εμφανίστηκε στην βάση δεδομένων &
Ναι \\ \hline
Εμφάνιση όλων των αναφορών στον χάρτη &
Ανάκτηση όλων των αναφορών με σωστές συντεταγμένες και οπτικοποίηση τους &
Οι αναφορές ανακτήθηκαν και εμφανίστηκαν στον χάρτη &
Ναι \\ \hline
Ενημέρωση κατάστασης αναφοράς από διαχειριστή &
Η αλλαγή ή διαγραφή κατάστασης αντικατοπτρίζεται άμεσα στον χάρτη &
Η αναφορά άλλαξε ή διαγράφτηκε και ενημερώθηκε ο χάρτης &
Ναι \\ \hline
Επαλήθευση αποστολής και απεικόνισης εικόνας &
Οι εικόνες αποθηκεύονται και συνδέονται σωστά με κάθε αναφορά &
Η εικόνα αποθηκεύτηκε σωστά στην βάση και εμφανίζεται στην διεπαφή όταν πατηθεί η αναφορά της &
Ναι \\ \hline
\end{tabular}
\end{table}

\subsection{Αξιολόγηση}
Η αξιολόγηση της πλατφόρμας, έχει σκοπό να εκτιμήσει την χρηστικότητα και την απόδοση της στην διαδικασία αναφοράς και διαχείρισης ζητημάτων ενός δήμου. Δεδομένου ότι το σύστημα βρίσκεται σε πειραματικό στάδιο και δεν έχει κυκλοφορηθεί δημόσια, η αξιολόγηση έγινε κυρίως μέσω επίδειξης.
\\
\\
Οι συμμετέχοντες παρακολούθησαν ένα περιεκτικό βίντεο παρουσίασης της εφαρμογής και της ιστοσελίδας, απεικονίζοντας την πλήρη λειτουργικότητα τους από την μεριά του χρήστη. Η επίδειξη παρουσίαζε την σύνδεση σε λογαριασμό, την δημιουργία μίας αναφοράς με φωτογραφία και περιγραφή μέσω της εφαρμογής, και την οπτικοποίηση της πάνω στον διαδραστικό χάρτη της ιστοσελίδας.
\\
\\
Το τελικό σύνολο του δείγματος έφτασε τις 27 απαντήσεις που συλλέχθηκαν μέσω ενός διαδικτυακού ερωτηματολογίου χρησιμοποιώντας την κλίμακα Likert με πέντε βαθμίδες, όπου το “1” να δηλώνει “Διαφωνώ Απόλυτα” και αντίστοιχα το “5” να σημαίνει “Συμφωνώ Απόλυτα”. Οι ερωτήσεις κάλυψαν τις βασικές πλευρές της εμπειρίας ενός χρήστη, όπως η ευκολία χρήσης, σαφήνεια της πλοήγησης, καλή ανταπόκριση και γενική ικανοποίηση.
\\
\\
Παρακάτω, παρουσιάζονται τα αποτελέσματα και η ανάλυση κάθε ερώτησης, συγκρίνοντας τις με το πως αυτά τα ευρήματα σχετίζονται με την πραγματική συμπεριφορά του συστήματος που παρατηρήθηκε κατά την υλοποίηση και τις τελικές δοκιμές.

\subsubsection*{Πληροφορίες Συμμετεχόντων}
Πριν τις βασικές ερωτήσεις, οι συμμετέχοντες ζητήθηκαν να παραχωρήσουν ορισμένες βασικές πληροφορίες όπως το φύλο και την ηλικία τους, αλλά και το ιστορικό τους σχετικά με την εξοικείωση τους, με την τεχνολογία και την πιθανή εμπειρία τους με την χρήση παρόμοιων εφαρμογών.
\\
\\
Όπως απεικονίζεται στον πίνακα \ref{tab:information}, η κατανομή των φύλων ήταν ισορροπημένη με 51,9\% των ερωτηθέντων να δηλώνουν “Άντρες” και 48,1\% να δηλώνουν “Γυναίκες”. Σχετικά με την ηλικία, η πλειοψηφία ήταν στο γκρουπ “Κάτω των 25” με 74,1\%, ενώ 11,1\% ήταν ανάμεσα σε 25 με 40 χρονών και το υπόλοιπο 14,8\% να αντιστοιχεί σε αυτούς πάνω των 40 ετών. Αυτή η κατανομή δείχνει πως το δείγμα αποτελούνταν κυρίως από νεαρούς συμμετέχοντες, τυπικά πιο εξοικειωμένους με την τεχνολογία, αν και η παρουσία μεγαλύτερων ηλικιών παρείχε πολύτιμες πληροφορίες.

\begin{table}[H]
\centering
\caption{Εισαγωγικές Πληροφορίες Χρηστών}
\label{tab:information}
\renewcommand{\arraystretch}{1.3}
\setlength{\tabcolsep}{8pt}
\begin{tabular}{|p{6cm}|p{3cm}|p{3cm}|}
\hline
\textbf{Κατηγορία} & \textbf{Γκρουπ} & \textbf{Ποσοστό (\%)} \\ \hline
\multirow{2}{*}{Φύλο} 
& Άνδρας & 51.9 \\ \cline{2-3}
& Γυναίκα & 48.1 \\ \hline
\multirow{3}{*}{Γκρουπ Ηλικίας} 
& Κάτω των 25 & 74.1 \\ \cline{2-3}
& 25–40 χρονών & 11.1 \\ \cline{2-3}
& Άνω των 40 & 14.8 \\ \hline
\end{tabular}
\end{table}

Οι συμμετέχοντες ακόμα, ερωτήθηκαν σε δύο εισαγωγικές αλλά σημαντικές πληροφορίες:

\begin{itemize}
    \item Την εμπειρία τους με την τεχνολογία (κλίμακα 1-5).
    \item Αν έχουν ξαναχρησιμοποιήσει παρόμοιες εφαρμογές.
\end{itemize}

Οι πιο πολλοί, βαθμολόγησαν τον εαυτό με την τιμή 4, υποδηλώνοντας μία γενική μέτρια προς θετική εξοικείωση. Όμως, το 63\% από αυτούς, δήλωσε πως δεν έχει ξαναχρησιμοποιήσει κάποιο παρόμοιο τύπο εφαρμογής, κάτι το οποίο σημαίνει πως αυτή η πλατφόρμα ήταν η πρώτη τους επαφή με τέτοιου είδους ψηφιακού συστήματος. Οι απαντήσεις τους εμφανίζονται αναλυτικά στον παρακάτω πίνακα:

\begin{table}[H]
\centering
\label{tab:tech_expirience}
\renewcommand{\arraystretch}{1.3}
\setlength{\tabcolsep}{7pt}
\begin{tabular}{|p{5.5cm}|p{3cm}|p{3cm}|}
\hline
\textbf{Ερώτηση} & \textbf{Απάντηση} & \textbf{Ποσοστό (\%)} \\ \hline
\multirow{5}{*}{Εξοικείωση στην τεχνολογία (1-5)} 
& 1 & 7.4 \\ \cline{2-3}
& 2 & 11.1 \\ \cline{2-3}
& 3 & 11.1 \\ \cline{2-3}
& 4 & 55.5 \\ \cline{2-3}
& 5 & 14.9 \\ \hline
\multirow{2}{*}{Εμπειρία με παρόμοια εφαρμογή} 
& Ναι & 37.0 \\ \cline{2-3}
& Όχι & 63.0 \\ \hline
\end{tabular}
\caption{Εξοικείωση με την τεχνολογία και προηγούμενη εμπειρία με παρόμοιες εφαρμογές}
\end{table}

\subsubsection*{Αποτελέσματα Ερωτηματολογίου}
\vspace{0.5cm}
\textit{Η εφαρμογή ήταν εύκολη στην κατανόηση και πλοήγηση.}

\begin{itemize}
    \item Βαθμολογία 5 (Συμφωνώ απόλυτα): 12
    \item Βαθμολογία 4 (Συμφωνώ): 14
    \item Βαθμολογία 3 (Ουδέτερος): 1
    \item Βαθμολογία 2 (Διαφωνώ): 0
    \item Βαθμολογία 1 (Διαφωνώ απόλυτα): 0
\end{itemize}

Η επικρατούσα τιμή είναι το 4 (Συμφωνώ), καθώς έχει την μεγαλύτερη συχνότητα (14).
Ο μέσος όρος είναι 4,4 και η διάμεσος 4.
Τα αποτελέσματα αυτά δείχνουν ότι σχεδόν όλοι οι συμμετέχοντες αντιλήφθηκαν την δομή και την ροή της εφαρμογής ως κατανοητή και εύχρηστη, κάτι το οποίο επιβεβαιώνει την επιτυχία του σχεδιασμού της διεπαφής του χρήστη.
(Σχήμα \ref{fig:question-1})
\begin{figure}[H]
\centering
\includegraphics[width=0.90\textwidth]{histograms/question-1.png}
\caption{Η εφαρμογή ήταν εύκολη στην κατανόηση και πλοήγηση.}
\label{fig:question-1}
\end{figure}

\newpage

\textit{Η διαδικασία δημιουργίας και αποστολής αναφοράς ήταν απλή.}

\begin{itemize}
    \item Βαθμολογία 5: 15
    \item Βαθμολογία 4: 11
    \item Βαθμολογία 3: 1
    \item Βαθμολογία 2: 0
    \item Βαθμολογία 1: 0
\end{itemize}

Η επικρατούσα τιμή είναι 5 (Συμφωνώ απόλυτα), με 15 εμφανίσεις.
Ο μέσος όρος είναι 4,5 και η διάμεσος 5.
Η υψηλή βαθμολογία δείχνει ότι οι χρήστες βρήκαν την διαδικασία υποβολής αναφοράς απλή και εύκολη, κάτι που διευκολύνει σημαντικά την συμμετοχή πολιτών.
(Σχήμα \ref{fig:question-2})

\begin{figure}[H]
\centering
\includegraphics[width=0.90\textwidth]{histograms/question-2.png}
\caption{Η διαδικασία δημιουργίας και αποστολής αναφοράς ήταν απλή.}
\label{fig:question-2}
\end{figure}

\vspace{0.5cm}

\textit{Ο σχεδιασμός και η αισθητική της εφαρμογής ήταν ελκυστικά.}

\begin{itemize}
    \item Βαθμολογία 5: 9
    \item Βαθμολογία 4: 12
    \item Βαθμολογία 3: 6
    \item Βαθμολογία 2: 0
    \item Βαθμολογία 1: 0
\end{itemize}

Η επικρατούσα τιμή είναι 4, με την μεγαλύτερη συχνότητα (12).
Ο μέσος όρος είναι 4,1 και η διάμεσος 4.
Οι περισσότεροι συμμετέχοντες έκριναν τον οπτικό σχεδιασμό ως ευχάριστο και καθαρό, γεγονός που ενισχύει την καλή πρώτη εντύπωση και την συνολική εμπειρία χρήσης.
(Σχήμα \ref{fig:question-3})

\begin{figure}[H]
\centering
\includegraphics[width=0.90\textwidth]{histograms/question-3.png}
\caption{Ο σχεδιασμός και η αισθητική της εφαρμογής ήταν ελκυστικά.}
\label{fig:question-3}
\end{figure}


\vspace{0.5cm}

\textit{Οι οδηγίες/μηνύματα της εφαρμογής ήταν σαφή και κατανοητά.}

\begin{itemize}
    \item Βαθμολογία 5: 11
    \item Βαθμολογία 4: 13
    \item Βαθμολογία 3: 3
    \item Βαθμολογία 2: 0
    \item Βαθμολογία 1: 0
\end{itemize}

Η επικρατούσα τιμή είναι 4.
Ο μέσος όρος είναι 4,3 και η διάμεσος 4.
Η πλειονότητα των απαντήσεων δείχνει ότι οι χρήστες κατανόησαν τα μηνύματα και τις οδηγίες χωρίς δυσκολία, στοιχείο που υποδηλώνει επιτυχημένη σχεδίαση επικοινωνίας με τον χρήστη.
(Σχήμα~\ref{fig:question-4})

\begin{figure}[H]
\centering
\includegraphics[width=0.90\textwidth]{histograms/question-4.png}
\caption{Οι οδηγίες/μηνύματα της εφαρμογής ήταν σαφή και κατανοητά.}
\label{fig:question-4}
\end{figure}

\vspace{0.5cm}

\textit{Η απεικόνιση των προβλημάτων στον χάρτη ήταν ευδιάκριτη και χρήσιμη.}

\begin{itemize}
    \item Βαθμολογία 5: 12
    \item Βαθμολογία 4: 11
    \item Βαθμολογία 3: 4
    \item Βαθμολογία 2: 0
    \item Βαθμολογία 1: 0
\end{itemize}

Η επικρατούσα τιμή είναι 5.
Ο μέσος όρος είναι 4,3 και η διάμεσος 4.
Η αξιολόγηση αυτή δείχνει ότι οι χρήστες θεωρούν τον χάρτη και την οπτικοποίηση του ως ένα λειτουργικό και ευδιάκριτο εργαλείο παρουσίασης των αναφορών, που συμβάλλει στην αποτελεσματική απεικόνιση των δεδομένων.
(Σχήμα~\ref{fig:question-5})

\begin{figure}[H]
\centering
\includegraphics[width=0.90\textwidth]{histograms/question-5.png}
\caption{Η απεικόνιση των προβλημάτων στον χάρτη ήταν ευδιάκριτη και χρήσιμη.}
\label{fig:question-5}
\end{figure}

\vspace{0.5cm}

\textit{Η συνολική εμπειρία χρήσης ήταν ευχάριστη και χωρίς δυσκολίες.}

\begin{itemize}
    \item Βαθμολογία 5: 15
    \item Βαθμολογία 4: 10
    \item Βαθμολογία 3: 2
    \item Βαθμολογία 2: 0
    \item Βαθμολογία 1: 0
\end{itemize}

Η επικρατούσα τιμή είναι 5.
Ο μέσος όρος είναι 4,5 και η διάμεσος 5.
Οι απαντήσεις καταδεικνύουν ότι οι περισσότεροι χρήστες θεώρησαν την εμπειρία ως θετική χωρίς τεχνικά προβλήματα ή καθυστερήσεις, επιβεβαιώνοντας την αξιοπιστία του συστήματος.
(Σχήμα~\ref{fig:question-6})

\begin{figure}[H]
\centering
\includegraphics[width=0.90\textwidth]{histograms/question-6.png}
\caption{Η συνολική εμπειρία χρήσης ήταν ευχάριστη και χωρίς δυσκολίες.}
\label{fig:question-6}
\end{figure}

\vspace{0.5cm}

\textit{Θα συνέχιζα να χρησιμοποιώ την εφαρμογή αν ήταν διαθέσιμη.}

\begin{itemize}
    \item Βαθμολογία 5: 19
    \item Βαθμολογία 4: 7
    \item Βαθμολογία 3: 1
    \item Βαθμολογία 2: 0
    \item Βαθμολογία 1: 0
\end{itemize}

Η επικρατούσα τιμή είναι 5.
Ο μέσος όρος είναι 4,7 και η διάμεσος 5.
Η σχεδόν ομόφωνη θετική πρόθεση δείχνει ότι οι χρήστες θα ήταν πρόθυμοι να συνεχίσουν την χρήση της εφαρμογής, γεγονός που αντικατοπτρίζει υψηλό επίπεδο ικανοποίησης.
(Σχήμα~\ref{fig:question-7})

\begin{figure}[H]
\centering
\includegraphics[width=0.90\textwidth]{histograms/question-7.png}
\caption{Θα συνέχιζα να χρησιμοποιώ την εφαρμογή αν ήταν διαθέσιμη.}
\label{fig:question-7}
\end{figure}

\vspace{0.5cm}

\textit{Θα πρότεινα την εφαρμογή σε άλλους.}

\begin{itemize}
    \item Βαθμολογία 5: 13
    \item Βαθμολογία 4: 10
    \item Βαθμολογία 3: 4
    \item Βαθμολογία 2: 0
    \item Βαθμολογία 1: 0
\end{itemize}

Η επικρατούσα τιμή είναι 5.
Ο μέσος όρος είναι 4,3 και η διάμεσος 4.
Η πρόθεση να προτείνουν την εφαρμογή σε άλλους χρήστες δείχνει υψηλό βαθμό αποδοχής και εμπιστοσύνης προς το σύστημα.
(Σχήμα~\ref{fig:question-8})

\begin{figure}[H]
\centering
\includegraphics[width=0.90\textwidth]{histograms/question-8.png}
\caption{Θα πρότεινα την εφαρμογή σε άλλους.}
\label{fig:question-8}
\end{figure}

\vspace{0.5cm}

\textit{Η εφαρμογή βοηθάει στη βελτίωση της επικοινωνίας πολιτών και δήμου.}

\begin{itemize}
    \item Βαθμολογία 5: 19
    \item Βαθμολογία 4: 8
    \item Βαθμολογία 3: 0
    \item Βαθμολογία 2: 0
    \item Βαθμολογία 1: 0
\end{itemize}

Η επικρατούσα τιμή είναι 5.
Ο μέσος όρος είναι 4,7 και η διάμεσος 5.
Η πλήρως θετική αξιολόγηση επιβεβαιώνει ότι οι χρήστες καταλήγουν πως η εφαρμογή μπορεί να γίνει ένα ουσιαστικό μέσο βελτίωσης της επικοινωνίας μεταξύ των πολιτών και δημοτικών αρχών, κάτι που ήταν σημαντικός στόχος της όλης πλατφόρμας.
(Σχήμα~\ref{fig:question-9})

\begin{figure}[H]
\centering
\includegraphics[width=0.90\textwidth]{histograms/question-9.png}
\caption{Η εφαρμογή βοηθάει στη βελτίωση της επικοινωνίας πολιτών και δήμου.}
\label{fig:question-9}
\end{figure}

\section{Συζήτηση των ευρημάτων}
Τα αποτελέσματα της αξιολόγησης  παρέχουν μία ολοκληρωμένη κατανόηση των αντιλήψεων που θα έχουν οι χρήστες σχετικά με την χρηστικότητα, την σχεδίαση και την συνολική λειτουργικότητα και αποδοτικότητα της πλατφόρμας. Η σταθερή παρουσία υψηλών βαθμολογιών σε όλες τις δηλώσεις δείχνει πως οι συμμετέχοντες συνολικά θεώρησαν το σύστημα ως αξιόπιστο και καλά δομημένο.
\\
\\
Η τιμή μέσου όρου 4,4 στο αν τους φάνηκε η εφαρμογή και η πλοήγηση της εύκολη, επιβεβαιώνει πως η διεπαφή είναι σχεδιασμένη με σκοπό το να γίνει οικεία γρήγορα. Παρομοίως, η διαδικασία καταχώρησης αναφοράς βαθμολογήθηκε υψηλά (μέσο όρο 4,5), αποδεικνύοντας πως ο μηχανισμός υποβολής θεωρήθηκε απλός και ταχύς. Αυτά τα ευρήματα, επικυρώνουν την εστίαση της εφαρμογής πάνω σε ένα σχεδιασμό με βάση τον χρήστη και τους απλούς τρόπους αλληλεπίδρασης.
\\
\\
Η οπτική και η αίσθηση της εφαρμογής επίσης αξιολογήθηκαν θετικά (4,1), με τους χρήστες να την θεωρούν ως καθαρή και ελκυστική. Ενώ το σκορ ήταν ελάχιστα χαμηλότερο από τα υπόλοιπα, παραμένει σε ένα δυνατό επίπεδο ικανοποίησης, χρειάζοντας πιθανά ένα μικρό περιθώριο βελτίωσης.
\\
\\
Όσον αφορά την επικοινωνία και την ανατροφοδότηση, παρατηρείται μία συνολική συμφωνία (4.3) πως τα μηνύματα της εφαρμογής ήταν ξεκάθαρα και πως η απεικόνιση των αναφορών στον χάρτη είναι απλή και εύκολα κατανοητή. Αυτά τα αποτελέσματα δείχνουν την επιτυχής αρχιτεκτονική της πλατφόρμας ως προς την ικανότητα να εμφανίζει πληροφορίες στον χρήστη.
\\
\\
Τέλος, η συνολική εμπειρία και η πρόθεση για μελλοντική χρήση, είχαν από τα μεγαλύτερα αποτελέσματα, με μέσους όρους 4,5 και 4,7 αντίστοιχα. Η θετική ανταπόκριση υποδηλώνει υψηλό επίπεδο ικανοποίησης και εμπιστοσύνης σε μία τέτοια πλατφόρμα. Επιπρόσθετα, οι δηλώσεις πως θα πρότειναν την εφαρμογή σε άλλους (4.3) και πως αυτή βοηθάει στην επικοινωνία του πολίτη με τον δήμο (4.7), έφτασαν πολύ υψηλά επίπεδα συμφωνίας, τονίζοντας το πόσο αναγνωρίζεται η αξία ενός τέτοιου συστήματος στον πραγματικό κόσμο.
\\
\\
Συνοψίζοντας, τα αποτελέσματα συνολικά επιβεβαιώνουν πως η πλατφόρμα εκπληρώνει θετικά τους κύριους στόχους της. Παρόλο που υπάρχει παρουσία μικρών διαφοροποιήσεων σε κάποια σημεία, όπως η οπτική, αυτό είναι φυσιολογικό και παρέχει χρήσιμες πληροφορίες για μελλοντικές αλλαγές. Ακόμα, η πλήρης απουσία αρνητικών απαντήσεων υποδηλώνει ότι το σύστημα λειτούργησε χωρίς να παρουσιαστούν λειτουργικά προβλήματα και πως το αποτέλεσμα αποδείχθηκε σύμφωνο με τις αρχικές εκτιμήσεις.