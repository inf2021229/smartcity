\section{Ανάλυση Απαιτήσεων Χρηστών}

\drop{H}{ανάπτυξη} μιας τέτοιας εφαρμογής όπου έχει να κάνει με τις έξυπνες πόλεις έχει κύρια απαίτηση να κατανοηθούν οι πραγματικές ανάγκες των χρηστών αυτής. Η εφαρμογή και ιστοσελίδα που παρουσιάζονται σε αυτή την εργασία, αποτελούνται από δύο βασικές κατηγορίες χρηστών, τους πολίτες και τους υπαλλήλους του δήμου. Ο ρόλος των πολιτών είναι να εντοπίζουν τυχόν προβλήματα και οι ίδιοι να αναφέρουν αυτά στον δήμο χρησιμοποιώντας την εφαρμογή μέσω του κινητού τους. Από την άλλη, οι υπάλληλοι μέσω της διαδικτυακής ιστοσελίδας έχουν την δουλειά να διαχειρίζονται τις αναφορές και να κάνουν αλλαγές σε αυτές ανάλογα με την κατάσταση του προβλήματος.
\\
\\
Στην σημερινή εποχή, η προσέγγιση προς τον σχεδιασμό τέτοιων συστημάτων έχει στραφεί κυρίως προς τον άνθρωπο και όχι απλά μόνο στο κομμάτι της τεχνολογίας. Όπως αναφέρει η μελέτη των \textcite{janoskova2021concept}, από μόνη της η τεχνολογία δεν μπορεί να σιγουρέψει την επιτυχία ενός τέτοιου συστήματος. Αυτά πρέπει να σχεδιαστούν με σκοπό να επιλύσουν τις καθημερινές ανάγκες ενός πολίτη, να είναι προσβάσιμα ως προς την λειτουργία και να ενισχύουν την σύνδεση και εμπιστοσύνη του πολίτη με του δήμου. Συνεπώς, αν δεν καλύπτουν τις ανάγκες των πολιτών και αν η χρήση τους θεωρείται δύσκολη τότε χάνουν τον σκοπό τους και επομένως εγκαταλείπονται από τους χρήστες. 
\\
\\
Οι βασικές ανάγκες για τους πολίτες είναι:
\begin{itemize}
  \item Να μπορούν εύκολα και γρήγορα να καταγράφουν ένα πρόβλημα χωρίς δυσκολίες
  \item Εύκολη επισύναψη φωτογραφίας και εγγραφή κειμένου
  \item Αυτόματη καταχώρηση τοποθεσίας της αναφοράς
  \item Τρόπος ενημέρωσης της κατάστασης της
\end{itemize}  

Αντίστοιχα, οι υπάλληλοι του δήμου χρειάζονται:
\begin{itemize}
    \item Ένα οργανωμένο περιβάλλον εργασίας όπου φαίνονται οι αναφορές
    \item Δυνατότητα διαχείρισης των καταχωρήσεων αυτών
    \item Να μπορούν να ενημερώνουν τις αναφορές
\end{itemize}

Άρα, σύμφωνα με όσα αναφέρθηκαν, η επιτυχία μιας τέτοιας εφαρμογής έξυπνης πόλης επηρεάζεται αρκετά από το πόσο απλά και ορθά εξυπηρετούνται και καλύπτονται οι ομάδες των πολιτών και των δήμων. Επιπλέον, είναι κατανοητό πως η εμπειρία του χρήστη (UX) είναι σημαντική τόσο για να γίνει η εφαρμογή αποδεκτή όσο και να διατηρήσει η ίδια ενεργή την χρήση της στο μέλλον. Οπότε, αν η όλη διαδικασία φαίνεται κουραστική και πολύπλοκη ως προς την χρήση, τότε όσο τέλεια τεχνολογικά και να είναι πάντα υπάρχει το ενδεχόμενο να μπορεί να αποτύχει στο τέλος.
\\
\\
Η πλατφόρμα που αναπτύχθηκε προσπαθεί να καλύψει όλες αυτές τις ανάγκες και απαιτήσεις που προαναφέρθηκαν. Η εφαρμογή για τα κινητά απλοποιεί αρκετά την διαδικασία της αναφοράς ενός προβλήματος με τρία απλά βήματα. Ο χρήστης αρχικά μέσω της εφαρμογής, φωτογραφίζει το σημείο του προβλήματος, προσθέτει μια σύντομη περιγραφή για αυτό και το σύστημα από μόνο του αποθηκεύει την γεωγραφική θέση της συσκευής του. Κατά την καταχώρηση, τα δεδομένα αυτά αποθηκεύονται και εμφανίζονται στην ιστοσελίδα της πλατφόρμας όπου παρουσιάζονται πάνω σε ένα χάρτη ως τοποθεσίες σε πραγματικό χρόνο. Οι υπεύθυνοι έχουν πρόσβαση σε αυτή την ιστοσελίδα και μπορούν να φιλτράρουν τις ενεργές καταχωρίσεις με βάση την κατάσταση τους και να τις διαχειριστούν ανάλογα με την πράξη του δήμου. Η ενημέρωση και επικοινωνία των δύο αυτών πλευρών γίνεται μέσω των αλλαγών πάνω στην κάθε αναφορά.
\\
\\
\newpage
Οι κύριες λειτουργίες που απαιτούνται για το σύστημα:
\begin{table}[H]
\centering
\caption{Λειτουργικές Απαιτήσεις.\label{tab:functional}}
\begin{tabular}{c p{11cm}}
\toprule
\textbf{} & \textbf{Απαίτηση} \\
\midrule
1   &   Οι πολίτες να μπορούν να υποβάλουν αναφορά προβλήματος με φωτογραφία και περιγραφή. \\
2   &   Η εφαρμογή καταγράφει αυτόματα τη γεωγραφική τοποθεσία της αναφοράς. \\
3   &   Οι πολίτες μπορούν να βλέπουν το ιστορικό και την κατάσταση των αναφορών τους. \\
4   &   Οι υπάλληλοι του δήμου έχουν πρόσβαση σε όλες τις αναφορές μέσω διαδικτυακής πλατφόρμας. \\
5   &   Οι υπάλληλοι μπορούν να εγκρίνουν, ενημερώνουν ή διαγράφουν αναφορές. \\
6   &   Το σύστημα αποθηκεύει όλες τις αναφορές σε ασφαλή βάση δεδομένων. \\
\bottomrule
\end{tabular}
\end{table}

Εκτός από αυτές τις λειτουργίες, το σύστημα πρέπει να πληροί ορισμένες απαιτήσεις προς την ποιότητα:
\begin{table}[H]
\centering
\caption{Ποιοτικές Απαιτήσεις.\label{tab:nonfunctional}}
\begin{tabular}{c p{11cm}}
\toprule
\textbf{} & \textbf{Απαίτηση} \\
\midrule
1 & Η εφαρμογή για κινητά πρέπει να λειτουργεί σε όλες τις συσκευές Android. \\
2 & Η διαδικτυακή πλατφόρμα πρέπει να είναι συμβατή σε όλους τους browser. \\
3 & Οι κλήσεις API πρέπει να έχουν γρήγορο χρόνο απόκρισης. \\
4 & Η ανταλλαγή δεδομένων πρέπει να γίνεται με ασφάλεια. \\
5 & Η διεπαφή χρήστη πρέπει να είναι απλή και κατανοητή. \\
\bottomrule
\end{tabular}
\end{table}
\newpage
Η προσέγγιση αυτή δεν έγινε τυχαία και στηρίζεται πάνω στο γεγονός πως οι πολίτες δεν είναι απλά παθητικοί δέκτες υπηρεσιών αλλά συμβάλλουν μέσω της ενεργής συμμετοχής τους πάνω στον σχεδιασμό του περιβάλλοντος όπου ζούνε εκείνοι. Η συλλογή δεδομένων από την ίδια την κοινότητα της πόλης για προβλήματα της όπως βλάβες, φώτα και λακκούβες μπορεί να καλυτερεύσει την εμπιστοσύνη ανάμεσα τους και να ενισχύσει την αποδοτικότητα του δήμου πάνω σε τέτοια θέματα. Τέλος, το πιο βασικό στοιχείο στην φάση της ανάλυσης των ζητούμενων μπορεί να θεωρηθεί ο ορισμός των ρόλων. Η μεριά του πολίτη έχει τον ρόλο να καταγράψει και να παρατηρεί τυχόν προβλήματα, ενώ η άλλη πλευρά του δήμου λειτουργεί ως διαχειριστής της όλης ενέργειας επεξεργάζοντας και ενημερώνοντας τις δηλώσεις. Η εφαρμογή οφείλει να σχεδιαστεί έτσι ώστε κάθε μια από τις πλευρές να γνωρίζει και να καταλαβαίνει τον σκοπό και την δουλειά της, κάτι που ταυτόχρονα διαφοροποιεί αλλά αυξάνει την αποτελεσματικότητα της διαδικασίας.
\\
\section{Επιλογή Τεχνολογιών}
Η επιλογή των κατάλληλων τεχνολογίων για αυτό το έργο ήταν σημαντική για την επίτευξη διατήρησης, επεκτασιμότητας και ασφάλειας. Το όλο σύστημα είναι φτιαγμένο από πολλά διαφορετικά μέρη, την κινητή εφαρμογή, την διαδικτυακή ιστοσελίδα, το backend server και την βάση δεδομένων σε cloud. Κάθε μέρος αποτελείται από τα δικά του εργαλεία όπου είναι αξιόπιστα και είναι συμβατά μεταξύ τους. Το τελικό μοντέλο συνδυάζει τεχνολογίες και ασφαλή διαχείριση δεδομένων κάτι που το κάνει ιδανικό για πλατφόρμα έξυπνης πόλης όπου χρειάζεται να είναι αποδοτικό και κατάλληλο προς την εμπειρία του χρήστη. Οι τεχνολογίες που επιλέχθηκαν και οι ρόλοι τους αναφέρονται στον Πίνακα \ref{tab:technologies}.

\begin{table}[H]
\centering
\caption{Οι τεχνολογίες που επιλέχθηκαν\label{tab:technologies}}
\begin{tabular}{p{4cm} p{10cm}}
\toprule
\textbf{Τεχνολογία} & \textbf{Σκοπός} \\
\midrule
Ionic React & Υβριδική ανάπτυξη εφαρμογής για κινητά. \\
Capacitor & Ενσωμάτωση τοπικών λειτουργιών της συσκευής (Κάμερα, GPS). \\
Leaflet.js & Οπτικοποίηση του χάρτη. \\
Node.js & Περιβάλλον διαχείρισης αιτημάτων API. \\
Express.js & Δημιουργία των RESTful APIs. \\
MongoDB Atlas & Βάση δεδομένων στο cloud για την αποθήκευση αναφορών και δεδομένων χρηστών. \\
TypeScript & Γλώσσα για καλύτερη συντήρηση και πρόληψη σφαλμάτων. \\
Bcrypt & Κρυπτογράφηση δεδομένων χρήστη. \\
Multer & Διαχείριση εικόνων για το backend. \\
CORS & Επιτρέπει με ασφάλεια τα API αιτήματα μεταξύ κινητού, ιστού και server. \\
\bottomrule
\end{tabular}
\end{table}

\hfill

\subsection{Αιτιολόγηση και Λεπτομέρειες}

\textbf{Ionic React, Capacitor, and Leaflet}

Η κινητή εφαρμογή υλοποιήθηκε με την χρήση του Ionic React για τις ικανότητες του πάνω στην ανάπτυξη πολλαπλών πλατφορμών. Το Capacitor συμπεριλήφθηκε ώστε να μπορεί να υπάρχει πρόσβαση σε δυνατότητες όπως κάμερα και τοποθεσία. Τέλος το Leaflet επιλέχθηκε για την προβολή των καταχωρήσεων πάνω σε περιβάλλον χάρτη. Αυτές οι επιλογές αναπτύχθηκαν περισσότερο στο Κεφάλαιο 2.

\hfill

\textbf{Node.js}

Το λειτουργικό κομμάτι του συστήματος είναι φτιαγμένο με το Node.js, ένα ασύγχρονο σύστημα εκτέλεσης της JavaScript όπου διαχειρίζεται αποτελεσματικά πολλαπλές συνδέσεις. Το Node είναι ευρέως αναγνωρισμένο για την αρχιτεκτονική του, καθιστώντας το κατάλληλο για εφαρμογές πραγματικού χρόνου που περιλαμβάνουν συχνές αλληλεπιδράσεις μέσω API \parencite{NodeJS}. 

\newpage

\textbf{Express.js}

Το Express.js, είναι ένα ελαφρύ framework που βασίζεται στο Node, γνωστό για την δημιουργία τελικών API σημείων και δομών δρομολόγησης \parencite{ExpressJS}. Ακόμα, προσφέρει ενδιάμεσα εργαλεία όπως το Cors και το Multer. Ο συνδυασμός του με το Node επιτρέπει την γρήγορη ανάπτυξη και διαχωρισμό των διαδρομών των λειτουργιών του συστήματος όπως η υποβολή αναφορών και η ανάκτηση των δεδομένων.

\hfill

\textbf{Mongo DB Atlas}

Η βάση δεδομένων εφαρμόστηκε μέσω του MongoDB Atlas, μιας cloud επιλογής όπου δεν χρησιμοποιεί SQL. Σε αντίθεση με άλλες παραδοσιακές βάσεις, το MongoDB αποθηκεύει τα δεδομένα σε αρχεία BSON όπου μοιάζουν με JSON αλλά επιτρέπουν ταχύτερη πρόσβαση σε αυτά. Αυτό είναι ιδιαίτερα χρήσιμο για την αποθήκευση αναφορών των χρηστών όπου αποτελούνται από διάφορα στοιχεία όπως εικόνα, κείμενο και γεωγραφικά στοιχεία. Το MongoDB εξασφαλίζει υψηλή διαθεσιμότητα και ασφάλεια κάνοντας το εξαιρετική επιλογή για υπηρεσία στο cloud \parencite{chauhan2019review}.

\hfill

\textbf{TypeScript}

Η λογική της εφαρμογής γράφτηκε σε TypeScript για την συντήρηση του κώδικα και την μείωση σφαλμάτων. Όπως αναφέρει ο \textcite{holmberg2023migrating}, η αλλαγή από JavaScript σε TypeScript προσφέρει σημαντικά πλεονεκτήματα. Κάποια από αυτά είναι η καλύτερη ανάγνωση του κώδικα και ο βελτιωμένος εντοπισμός σφαλμάτων κατά την ανάπτυξη. Αυτά τα οφέλη είναι απαραίτητα για μεγάλα έργα όπου προτεραιότητα είναι η δομή και η αξιοπιστία.

\hfill

\textbf{Bcrypt}

Η ασφάλεια ήταν μια σημαντική ανησυχία ειδικά στην διαχείριση πληροφοριών των χρηστών. Το Bcrypt χρησιμοποιήθηκε για την κρυπτογράφηση με κατακερματισμό των αποθηκευμένων κωδικών πρόσβασης, εγγυώντας πως αυτοί θα παραμείνουν ασφαλείς. Ο κατακερματισμός μειώνει τον κίνδυνο επιθέσεων brute-force και άλλων μεθόδων. Έτσι το Bcrypt μπορεί να θεωρηθεί μια εξαιρετική λύση για την ασφάλεια ελέγχου ταυτότητας \parencite{sriramya2015providing}.

\newpage

\textbf{Multer}

Το Multer είναι ένα ενδιάμεσο λογισμικό του Node.js όπου υποστηρίζει δεδομένα πολλαπλών τμημάτων \parencite{Multer}. Αυτό χρειάζεται για την διαχείριση των αρχείων όπως το ανέβασμα των φωτογραφιών από την κινητή εφαρμογή καθώς τις επεξεργάζεται και τις αποθηκεύει για τις επόμενες λειτουργίες του backend.

\hfill

\textbf{CORS}

Το CORS είναι ένα πακέτο του Node όπου λειτουργεί σαν ενδιάμεσο του Express όπου επιτρέπει την ασφαλή κοινή χρήση μεταξύ προελεύσεων, αποτρέποντας προβλήματα ασφαλείας από τα προγράμματα περιήγησης όταν ζητούν πρόσβαση στο API \parencite{Cors}.
\\
\section{Αρχιτεκτονική Συστήματος}

Η πλατφόρμα όπου αναπτύχθηκε για αυτή την εργασία ακολουθεί μία αρχιτεκτονική τριών επιπέδων. Αυτή η απόφαση να διαχωριστούν σε μικρότερα μέρη έγινε για την βελτίωση επεκτασιμότητας και συντήρησης των κώδικων. Έχοντας τις λειτουργίες της πλατφόρμας σε διαφορετικά συστήματα, έγινε ευκολότερη η ενημέρωση και η τροποποίηση οποιουδήποτε κομματιού χωρίς να επηρεάζει άμεσα τα υπόλοιπα. 
\\
\\
Η κύρια ιδέα της σχεδίασης είναι ότι οι βασικοί χρήστες, όπου είναι οι πολίτες, αλληλεπιδρούν μέσω της εφαρμογής για κινητά, ενώ η άλλη ομάδα χρηστών, οι υπάλληλοι των δήμων, χρησιμοποιούν την διαδικτυακή ιστοσελίδα. Αυτές οι δύο διεπαφές επικοινωνούν με ένα backend server όπου διαχειρίζεται την ροή δεδομένων και τις λειτουργίες της βάσης. Το server έχει τον ρόλο να βοηθάει στην μεταφορά των πληροφοριών μεταξύ των δύο ομάδων και της βάσης, είτε αυτές υποβάλλουν είτε λαμβάνουν στοιχεία.

\begin{figure}[H]
\centering
\includegraphics[width=0.60\textwidth]{images/system_architecture2.jpg}
\caption{Αρχιτεκτονική της Πλατφόρμας Αναφοράς Έξυπνων Πόλεων}
\label{fig:system_architecture}
\end{figure}

\hfill

\subsection{Κινητή Εφαρμογή}

Η κινητή εφαρμογή αποτελεί το κύριο εργαλείο για τους κατοίκους. Τους δίνει την δυνατότητα να βγάλουν μια φωτογραφία από ένα πρόβλημα, όπως μια λακκούβα ή ένα σπασμένο φως δρόμου, να γράψουν μια σύντομη περιγραφή και να καταχωρήσουν την αναφορά τους. Κατά την καταχώρηση η γεωγραφική τοποθεσία του προβλήματος λαμβάνεται αυτόματα μέσω της συσκευής. Αυτές οι τρεις πληροφορίες υποβάλλονται στην βάση δεδομένων με το πάτημα ενός κουμπιού.
\\
\\
Η συνολική λειτουργία μπορεί να ακούγεται απλή αλλά οι τεχνολογίες πίσω που την κάνουν δυνατή είναι σημαντικές. Η εφαρμογή δημιουργήθηκε με την χρήση του framework Ionic και της React. Η ίδια εκτελείται σε TypeScript για την μείωση σφαλμάτων και χρησιμοποιεί το Capacitor για την πρόσβαση στις υπηρεσίες της συσκευής που χρειάζεται για να λειτουργεί όπως η κάμερα, GPS και την δυνατότητα ανοίγματος των ρυθμίσεων μέσω εφαρμογής για την πιθανή ενεργοποίηση δικαιωμάτων χειροκίνητα.

\newpage

Όλες οι πληροφορίες όπου συλλέγονται από την εφαρμογή στέλνονται στο backend μέσω της χρήσης της επιλογής αιτήματος για ποστάρισμα (POST) του API. Η κάθε καταχώρηση αποτελείται από την εικόνα, την περιγραφή κειμένου και το γεωγραφικό πλάτος και μήκος της. Ο χρήστης δεν εμπλέκεται με το κομμάτι της αλληλεπίδρασης με την βάση απευθείας, η εφαρμογή απλά επικοινωνεί με το server και αυτό αναλαμβάνει τις υπόλοιπες ενέργειες.

\subsection{Server}
Το πίσω μέρος της πλατφόρμας είναι το πιο σημαντικό. Κατασκευάστηκε με την χρήση του Node.js με το framework Express.js. Αυτό το κομμάτι του συστήματος είναι υπεύθυνο για την διαχείριση της λογικής του και την μεταφορά των δεδομένων που υποβάλλονται από τους χρήστες. Η δουλειά του είναι να λαμβάνει όλα τα αιτήματα από την κινητή εφαρμογή και την διαδικτυακή ιστοσελίδα και να απαντάει αντίστοιχα σε κάθε ένα.
\\
\\
Το server αποτελείται από ένα σύνολο τελικών σημείων API. Για παράδειγμα:
\begin{itemize}
    \item Όταν οι χρήστες καταχωρούν μία αναφορά, η εφαρμφογή στέλνει ένα αίτημα “POST” στο σημείο “/api/report”.
    \item Όταν η ιστοσελίδα χρειάζεται να ανακτήσει όλα τα δεδομένα, στέλνει ένα αίτημα ανάκτησης “GET” στο αντίστοιχο “/api/report” σημείο.
\end{itemize}

Με την χρήση του Express είναι εύκολη η οργάνωση και σύνδεση αυτών των διαδρομών. Ακόμα, ξεκλειδώνει την πρόσβαση σε δικά του εργαλεία όπως το Multer και το CORS. Το Multer χρησιμοποιείται για την διαχείριση των φωτογραφιών που στέλνονται από την εφαρμογή, οι οποίες μετά μεταφέρονται στην βάση. Το CORS χρησιμοποιείται ώστε οι δύο διεπαφές του συστήματος να μπορούν να επικοινωνούν με το backend παρόλο που τρέχουν ξεχωριστά μεταξύ τους.
\\
\\
Η ασφάλεια γίνεται μέσω του Bcrypt όπου κρυπτογραφεί τους κωδικούς πρόσβασης των χρήστών. Η κρυπτογράφηση γίνεται μέσω κατακερματισμού με τυχαία σειρά χαρακτήρων και διασφαλίζει πως οι τελικοί κωδικοί που θα αποθηκευτούν στην βάση δεδομένων δεν θα είναι αναγνώσιμοι. 
\\
\\
Όλα τα δεδομένα αποθηκεύονται στην cloud βάση MongoDB Atlas. Το MongoDB βασίζεται πάνω σε αρχεία οπότε μπορεί να αποθηκεύσει δεδομένα σε διάφορες δομές. Αυτό κάνει την αποθήκευση των αναφορών πιο εύκολη αφού συμπεριλαμβάνουν πολλά διαφορετικά πεδία.

\subsection{Βάση Δεδομένων}
Η βάση δεδομένων υλοποιείται στην MongoDB Atlas με δύο κύριες συλλογές, τους users και τα reports. Η συλλογή users αποθηκεύει στοιχεία λογαριασμού όπως email, κωδικό, τιμή που προσδιορίζει αν ο χρήστης είναι διαχειριστής και το μοναδικό του id. Η συλλογή reports καταγράφει κάθε αναφορά με ένα id αυτής, το id του χρήστη που την δημιούργησε, μία περιγραφή, την εικόνα, τα γεωγραφικά στοιχεία και την κατάσταση της ως ακέραιο με τιμές 0, 1 και 2. Η σχέση μεταξύ των δύο συλλογών είναι 1:N (ένας χρήστης : πολλές αναφορές) και η σύγκριση των id των χρηστών όπως φαίνεται και στο σχήμα \ref{fig:database}, ελέγχεται στο επίπεδο του backend, ώστε να διασφαλίζεται ότι κάθε αναφορά αντιστοιχίζεται σε υπαρκτό χρήστη.
\\
\\
Το σχήμα παραμένει απλό για γρήγορες εγγραφές και άμεση απεικόνιση στον χάρτη. Η επιλογή ξεχωριστών πεδίων πλάτους και μήκους απλοποιεί την καταχώριση από την κινητή εφαρμογή, ενώ η αποθήκευση εικόνας ως base64 διευκολύνει την ενιαία ροή αποστολής της μέσω του API.
\\
\begin{figure}[H]
\centering
\includegraphics[width=0.70\textwidth]{images/database.jpg}
\caption{Συλλογές users και reports της Βάσης Δεδομένων}
\label{fig:database}
\end{figure}

\subsection{Ιστοσελίδα}
Το τρίτο μέρος της πλατφόρμας είναι η ιστοσελίδα του δήμου. Η διεπαφή αυτή είναι δημοσίως προσβάσιμη και εξυπηρετεί και τις δύο ομάδες χρηστών. Οι πολίτες μπορούν να την χρησιμοποιήσουν για την προβολή των ενεργών καταχωρήσεων πάνω σε χάρτη. Έτσι, οι ίδιοι παραμένουν ενημερωμένοι για τα υπάρχοντα προβλήματα στην πόλη τους και μπορούν να βλέπουν αν έχουν ήδη καταχωρηθεί συγκεκριμένες αναφορές χωρίς την χρήση της εφαρμογής. Στην άλλη πλευρά, οι υπάλληλοι του δήμου έχουν πρόσβαση στην ίδια διεπαφή. Εκτός την προβολή των καταχωρήσεων μπορούν να ενημερώσουν την κατάσταση προόδου και να τις σβήσουν όταν είναι πλέον φτιαγμένα μέσω μία σελίδας διαχειριστή. 
\\
\\
Η ιστοσελίδα είναι φτιαγμένη χρησιμοποιώντας HTML, CSS και JavaScript. Ο χάρτης υλοποιήθηκε με την χρήση της βιβλιοθήκης Leaflet για διάδραση. Ενώ φορτώνει η σελίδα, στέλνεται απευθείας ένα αίτημα στο server για την ανάκτηση των δεδομένων από τις καταχωρήσεις. Μόλις λάβει αυτές τις πληροφορίες, το Leaflet εμφανίζει κάθε αναφορά ως ένα σημείο πάνω στον χάρτη βασισμένο στις συντεταγμένες της. 
\\
\\
Κάθε σημείο περιλαμβάνει την εικόνα, το κείμενο και την κατάσταση της αντίστοιχης αναφοράς όπου εμφανίζονται πατώντας πάνω του. Επίσης, οι δείκτες έχουν διαφορετικό χρώμα αναλόγως την πρόοδο του με κόκκινο να σημαίνει πως το πρόβλημα είναι νέο και δεν έχει γίνει κάποια ενέργεια πάνω του ακόμα, πορτοκαλί για να δείξει ότι είναι σε εξέλιξη η διαδικασία επίλυσης του και πράσινο όταν έχει πλέον διορθωθεί. Η επεξεργασία της κατάστασης των αναφορών γίνεται μέσω αιτήματος διόρθωσης (PATCH) στο API και αντίστοιχα αυτές σβήνονται μέσω αιτήματος διαγραφής (DELETE). 

\subsection{Τελικό Αποτέλεσμα}
Κάθε κομμάτι του συστήματος δουλεύει ανεξάρτητα με το άλλο αλλά όλα επικοινωνούν μεταξύ τους ομαλά μέσω του API. Η εφαρμογή για κινητά στέλνει δεδομένα, το backend τα στέλνει για αποθήκευση στην βάση δεδομένων και τα επεξεργάζεται, και η ιστοσελίδα τα οπτικοποιεί. Χρησιμοποιώντας αυτή την προσέγγιση, το αποτέλεσμα αφήνει ελεύθερο κάθε μέρος να εξελίσσεται μόνο του. Για παράδειγμα, η εφαρμογή μπορεί να ενημερωθεί ώστε να υποστηρίζει και iOS στο μέλλον χωρίς να επηρεάσει τις υπόλοιπες λειτουργίες. Άρα, κάθε κομμάτι επικεντρώνεται πάνω σε συγκεκριμένες ανάγκες χωρίς να βρίσκεται εμπλεκόμενο με τα υπόλοιπα ή να δημιουργεί συγκρούσεις μεταξύ τους.

\section{Διαγράμματα Χρήσης και Ροής}
Για να γίνει καλύτερα κατανοητό πως τα διαφορετικά μέρη της πλατφόρμας αλληλεπιδρούν το ένα με το άλλο και πως οι χρήστες χρησιμοποιούν το σύστημα, δύο είδη διαγραμμάτων έχουν δημιουργηθεί, τα διαγράμματα περίπτωσης χρήσης και ροής του συστήματος. Αυτά βοηθούν στην προβολή των διαφορετικών ρόλων στο σύστημα και τα βήματα όπου ακολουθούνται κατά την χρήση κάθε συστήματός.

\subsection{Διάγραμμα Περίπτωσης Χρήσης}
Το σύστημα αποτελείται από δύο ομάδες χρηστών, τους πολίτες και τις αρχές. Οι πολίτες είναι υπεύθυνοι για τον εντοπισμό προβλημάτων στην πόλη. Η αλληλεπίδραση τους γίνεται μέσω της κινητής εφαρμογής. Αυτοί μπορούν να:

\begin{itemize}
    \item Βγάλουν φωτογραφία το πρόβλημα
    \item Γράψουν μία σύντομη περιγραφή
    \item Καταγράψουν αυτόματα την τοποθεσία τους
    \item Καταχωρήσουν την αναφορά
    \item Βλέπουν τις καταστάσεις των αναφορών τους
\end{itemize}

Από την άλλη, οι υπάλληλοι μέσα από την ιστοσελίδα επιβλέπουν και διαχειρίζονται τις αναφορές. Οι χρήσεις που έχουν είναι να:

\begin{itemize}
    \item Βλέπουν όλες τις αναφορές
    \item Ενημερώσουν την κατάσταση μίας αναφοράς
    \item Διαγράψουν αναφορές που έχουν ολοκληρωθεί ή είναι λάθος
\end{itemize}

\hfill

Όπως φαίνεται και στο Διάγραμμα Περίπτωσης Χρήσης (Σχήμα \ref{fig:use_case}), κάθε πλευρά είναι συνδεδεμένη στις ενέργειες που επιτρέπονται να εκτελέσουν. Αυτό δείχνει πως έχουν χωριστεί οι άδειες και πόσο διαφορετικές είναι οι δυνατότητες όπου έχουν οριστεί σε κάθε είδος χρήστη.

\begin{figure}[H]
\centering
\includegraphics[width=0.70\textwidth]{images/use_case.jpg}
\caption{Διάγραμμα Περίπτωσης Χρήσης}
\label{fig:use_case}
\end{figure}

\subsection{Διαγράμματα Ροής}
\subsubsection*{Διάγραμμα Εφαρμογής}
Το διάγραμμα ροής της κινητής εφαρμογής (Σχήμα \ref{fig:flow-mobile}), δείχνει την πλήρη διαδικασία που θα ακολουθήσει ένας κάτοικος κατά την χρήση της. Ανοίγοντας την εφαρμογή, εμφανίζεται μία φόρμα σύνδεσης του χρήστη. Αν ο χρήστης δεν έχει ένα λογαριασμό, μπορούν να τον δημιουργήσουν απευθείας στην ίδια σελίδα. Αφού δημιουργήσει ή αν έχει ήδη ένα λογαριασμό, τότε συνδέεται κανονικά και μεταφέρεται στην αρχική σελίδα. Σε αυτή, του παρέχονται δύο βασικές επιλογές, να αναφέρει ένα πρόβλημα ή να δει τις αναφορές του. Αν ο χρήστης διαλέξει να αναφέρει ένα πρόβλημα, κατευθύνεται στην σελίδα αναφοράς, όπου μπορεί να βγάλει μία φωτογραφία, να γράψει μία περιγραφή και να υποβάλει το θέμα στο σύστημα. Αντίθετα, διαλέγοντας το κουμπί “View My Reports”, ανοίγει η σελίδα των δηλώσεων του ή και όλων των υπαρχόντων αναφορών. Αυτή η ροή, τελειώνει με τον χρήστη να γυρνάει στην αρχική σελίδα και να κλείνει την εφαρμογή.
\\
\begin{figure}[H]
\centering
\includegraphics[width=0.55\textwidth]{images/flow-mobile.jpg}
\caption{Διάγραμμα Ροής Εφαρμογής}
\label{fig:flow-mobile}
\end{figure}

\subsubsection*{Διάγραμμα Ιστοσελίδας}
Για την ιστοσελίδα, υπάρχουν δύο ρόλοι χρήστη μέσα στο διάγραμμα ροής (Σχήμα \ref{fig:flow-web}), ο απλός χρήστης και ο διαχειριστής, οι οποίοι έχουν αντίστοιχα την δικιά τους διαφορετική αλληλεπίδραση σε αυτή. Οι απλοί επισκέπτες εισέρχονται στην ανοιχτή σελίδα του χάρτη, όπου μπορούν να διαλέξουν μία αναφορά και να δουν τις πληροφορίες της. Από την άλλη, οι διαχειριστές εισέρχονται στην κλειστή σελίδα διαχειριστή, στην οποία θα πρέπει να συνδεθούν με τον λογαριασμό τους. Αφού γίνει η σύνδεση, εμφανίζεται ο λίστα ελέγχου που δείχνει όλες τις ενεργές καταχωρήσεις. Από εκεί, μπορούν είτε να ενημερώσουν την κατάσταση μίας αναφοράς ή να την διαγράψουν τελείως. Η λίστα μετά διαμορφώνεται αντίστοιχα με την επιλογή του. Η ροή και των δύο φτάνει στο τέλος της όταν αυτοί κλείσουν την ιστοσελίδα.
\\
\begin{figure}[H]
\centering
\includegraphics[width=0.55\textwidth]{images/flow-web.jpg}
\caption{Διάγραμμα Ροής Ιστοσελίδας}
\label{fig:flow-web}
\end{figure}

\section{Σχεδιασμός UI/UX}
\subsection{Σχεδίαση Διεπαφής}
Η σχεδίαση της διεπαφής της πλατφόρμας είναι βασισμένη στην απλότητα και σαφήνεια σε όλα τα κομμάτια της. Αφού ο κύριος στόχος του συστήματος είναι να αφήνει τους κατοίκους να αναφέρουν θέματα της πόλης τους εύκολα και τους υπαλλήλους του δήμου να τα διαχειρίζονται αποδοτικά, το σχέδιο έπρεπε να παραμείνει ευθύ και λειτουργικό. Κάθε σελίδα χτίστηκε ώστε να είναι κατανοητή και μινιμαλιστική για την ολοκλήρωση της λειτουργίας της.
\\
\\
Κατά την όλη διάρκεια, ακολουθήθηκε μία λογική προτεραιότητας στα κινητά επειδή η εφαρμογή είναι το βασικό εργαλείο που θα χρησιμοποιηθεί περισσότερο από όλα. Η διεπαφή της εστιάζει στην απευθείας πρόσβαση στις κύριες ενέργειες, οι οποίες είναι η σύνδεση χρήστη, η δημιουργία αναφοράς και η προβολή της. Η σελίδα σύνδεσης επιτρέπει στον χρήστη να συνδεθεί με τον λογαριασμό του γρήγορα, ξεκλειδώνοντας του τις υπόλοιπες σελίδες. Η σελίδα αναφοράς παρέχει ένα απλό τρόπο για τον χρήστη να καταχωρήσει ένα πρόβλημα χρησιμοποιώντας την κάμερα και το GPS, και η σελίδα της λίστας των καταχωρήσεων αναφέρει όλες τις δηλώσεις του χρήστη αυτού ή και όλες τις διαθέσιμες αν το θελήσει ο ίδιος. Για να εισαχθεί ένας χρήστης σε αυτές τις δύο σελίδες, υπάρχει μία ενδιάμεση σελίδα όπου παρέχει την είσοδο μέσω της χρήσης κουμπιών με την σχετική ετικέτα. Αυτή η διαδικασία σχεδιάστηκε ώστε η αποστολή μίας αναφοράς να γίνεται σε σύντομο χρονικό διάστημα χωρίς να καθυστερεί τον χρήστη.
\\
\begin{figure}[H]
  \centering
  \tabcolsep=0.2\linewidth
  \divide\tabcolsep by 8
  \begin{tabular}{cccc}
    \includegraphics[width=0.17\linewidth]{images/login.jpg} &
    \includegraphics[width=0.17\linewidth]{images/home.jpg} &
    \includegraphics[width=0.17\linewidth]{images/report.jpg} &
    \includegraphics[width=0.17\linewidth]{images/myreports.jpg} \\
    \small (a) Σελίδα Σύνδεσης &
    \small (b) Αρχική Σελίδα &
    \small (c) Σελίδα Αναφοράς &
    \small (d) Σελίδα Καταχωρήσεων
  \end{tabular}
  \caption{Σελίδες της κινητής εφαρμογής}
  \label{fig:app_screens}
\end{figure}

Η ιστοσελίδα, όπου χρησιμοποιείται κυρίως από τους υπαλλήλους, εστιάζει κυρίως στην οπτικοποίηση των δεδομένων και την επεξεργασία τους. Ο χάρτης χρησιμεύει ως ένα μέσο για να γίνει αυτή η οπτικοποίηση, όπου τα σημάδια πάνω του εκπροσωπούν τις αναφορές στις διάφορες καταστάσεις. Κάθε κατάσταση έχει το δικό της χρώμα, βοηθώντας τους διαχειριστές στην σελίδα διαχειριστή, αλλά και τους χρήστες, να μπορούν να τα εντοπίσουν άμεσα και εύκολα. Ακόμα, στην βασική σελίδα του χάρτη και στην σελίδα διαχειριστή υπάρχουν μετρητές κάθε κατάστασης αναφοράς δίνοντας μία γρήγορη επισκόπηση χωρίς την ανάγκη επιπλέον πλοηγήσεων. Η ίδια η σελίδα διαχειριστή, είναι δομημένη έτσι ώστε να εμφανίζει κάθε αναφορά ξεχωριστά σε μορφή λίστας και του παρέχει επιλογές διαχείρισης ή διαγραφής της.
\\
\begin{figure}[H]
  \centering
  \tabcolsep=0.2\linewidth
  \divide\tabcolsep by 8
  \begin{tabular}{cccc}
    \includegraphics[width=0.50\linewidth]{images/map-page1.png} &
    \includegraphics[width=0.50\linewidth]{images/map-page2.png}\\
    \small (a) Σελίδα Χάρτη &
    \small (b) Αναφορά στον Χάρτη
  \end{tabular}
  \caption{Σελίδα Χάρτη της Ιστοσελίδας}
  \label{fig:website_screens1}
\end{figure}

\begin{figure}[H]
  \centering
  \tabcolsep=0.2\linewidth
  \divide\tabcolsep by 8
  \begin{tabular}{cccc}
    \includegraphics[width=0.50\linewidth]{images/admin-page1.png} &
    \includegraphics[width=0.50\linewidth]{images/admin-page2.png}\\
    \small (a) Σύνδεση Σελίδας Διαχειριστή &
    \small (b) Λίστα Ελέγχου
  \end{tabular}
  \caption{Σελίδα Διαχειριστή της Ιστοσελίδας}
  \label{fig:website_screens2}
\end{figure}

Η συνολική οπτική ταυτότητα είναι μοντέρνα και μινιμαλιστική. Η παλέτα χρωμάτων αποτελείται κυρίως από γαλάζιο και άσπρο διατηρώντας μία καθαρή και ουδέτερη εμφάνιση στην εφαρμογή, με τα γράμματα να είναι μαύρα για αντίθεση. Τα κουμπιά είναι στρογγυλεμένα, οι σκιές είναι απαλές κρατώντας το μοντέρνο θέμα της. Τα κείμενα και οι ετικέτες είναι σύντομα και κατανοητά για να παραμείνει η διεπαφή προσβάσιμη σε όλους τους χρήστες. Τέλος, χρησιμοποιήθηκαν σταθερές αποστάσεις και ευθυγραμμίσεις ώστε να διασφαλιστεί η αναγνωσιμότητα και να αποφευχθεί η οπτική πολυπλοκότητα κατά την χρήση.

\subsection{Εμπειρία Χρήστη και Ροή}
Ο σχεδιασμός της εμπειρίας χρήστη (UX) είναι στοχευμένος στο να κάνει τις αλληλεπιδράσεις όσο πιο το δυνατόν πιο διαισθητικές και αυτονόητες. Ο χρήστης θα πρέπει να βρεθεί σε θέση να καταλάβει πως να λειτουργεί την εφαρμογή απευθείας χωρίς την ανάγκη προηγούμενων οδηγιών. Για παράδειγμα, κατά την δημιουργία μίας νέας αναφοράς, η εφαρμογή αυτόματα ανιχνεύει την τοποθεσία του χρήστη μέσω του GPS, μειώνοντας την ανάγκη για χειροκίνητης εισαγωγής από αυτόν. Μετά την υποβολή, ένα μήνυμα επιβεβαίωσης εμφανίζεται για να διαβεβαιώσει τον χρήστη πως η ενέργεια εκτελέστηκε επιτυχώς και τον γυρνάει αυτόματα στην σελίδα του βασικού μενού. Ακόμα, προστέθηκαν δείκτες φόρτωσης και σύντομα μηνύματα σφαλμάτων ώστε να δέχεται άμεση ανατροφοδότηση κατά την διάρκεια των λειτουργιών. Η ιστοσελίδα ακολουθεί, επίσης, παρόμοια απλά μοτίβα αλληλεπίδρασης ώστε να παραμείνει η εμπειρία ομοιόμορφη και για τις δύο ομάδες χρηστών, τους πολίτες και τους διαχειριστές.
\\
\\
Η πλοήγηση μεταξύ σελίδων της εφαρμογής έγινε με το σύστημα δρομολόγησης της Ionic, επιτρέποντας για ομαλές μεταβάσεις και διατηρώντας την συμπεριφορά της προς τα πίσω κίνησης στις συσκευές. Αυτό διασφαλίζει συνοχή μεταξύ Android και άλλων πλατφορμών.
\\
\\
Συνολικά, η UI/UX σχεδίαση επιτυγχάνει μία ισορροπία μεταξύ χρηστικότητας και απλότητας. Εστιάζοντας στον μινιμαλισμό και σε σταθερά οπτικά στοιχεία, η πλατφόρμα παρέχει μία ευχάριστη εμπειρία και εμφάνιση και για τους πολίτες και για τους υπαλλήλους του δήμου.
\\
\\
Το ακόλουθο σχήμα \ref{fig:ui-wireflow} απεικονίζει την κύρια δομή πλοήγησης της εφαρμογής μέσω ενός wireflow διαγράμματος, δείχνοντας την σύνδεση μεταξύ των σελίδων και την λογική ροή των ενεργειών του χρήστη.
\\
\begin{figure}[H]
\centering
\includegraphics[width=0.85\textwidth]{images/ui-wireflow.jpg}
\caption{Wireflow Διεπαφής Χρήστη Εφαρμογής}
\label{fig:ui-wireflow}
\end{figure}

\section{Στρατηγικές Ανάπτυξης}
Η ανάπτυξη της πλατφόρμας δεν βασίστηκε πάνω σε μία μοναδική μεθοδολογία. Κάθε μέρος ακολούθησε την στρατηγική όπου ταίριαζε στο ρόλο του και τις τεχνικές ανάγκες του καλύτερα. Η λογική της κινητής εφαρμογής, της ιστοσελίδας και του backend υλοποιήθηκαν χρησιμοποιώντας μία Agile προσέγγιση, όπου τα χαρακτηριστικά τους, όπως σελίδες και λειτουργίες, δημιουργήθηκαν και δοκιμάστηκαν σε μικρά βήματα και βελτιώθηκαν σταδιακά με βάση τα αποτελέσματα. Αντίθετα, το σχήμα της βάσης δεδομένων ακολούθησε μία πιο δομημένη προσέγγιση όπως αυτή της στρατηγικής Καταρράκτη. Η επιλογή αυτή έγινε επειδή η σχεδίαση της βάσης έπρεπε να γίνει με σαφήνεια από την αρχή και να παραμείνει σταθερή κατά την διάρκεια του έργου.

\subsection{Agile}
Η στρατηγική που ακολουθήθηκε κατά την ανάπτυξη της λογικής των διεπαφών ιστού και κινητού καθώς και του backend, είναι αυτή της Agile. Η κύρια ιδέα πίσω από την Agile είναι η λειτουργία σε σύντομους κύκλους ανάπτυξης, όπου μικρά κομμάτια του συστήματος δημιουργούνται, δοκιμάζονται και βελτιώνονται πριν την μετάβαση στο επόμενο βήμα. Αυτό ταιριάζει λογικά με την πλατφόρμα, αφού επέτρεπε για κάθε νέα σελίδα, λειτουργία καθώς και τελικά σημεία στο API να επιβεβαιωθεί η σωστή λειτουργικότητα τους στην πράξη και να υλοποιηθούν πλήρως πριν την ένταξη επόμενων.

\begin{figure}[H]
\centering
\includegraphics[width=0.55\textwidth]{images/agile.png}
\caption{Στρατηγική Προσέγγισης Agile}
\label{fig:agile}
\end{figure}

Παρόλο που υπήρχε ένα ξεκάθαρο πλάνο από την αρχή σχετικά με την αρχιτεκτονική του συστήματος και τις κύριες λειτουργίες του, η υλοποίηση ολοκληρώθηκε βήμα βήμα σε επαναλαμβανόμενους κύκλους. Για παράδειγμα, η σύνδεση χρήστη, ενσωμάτωση της κάμερας, η απεικόνιση του χάρτη και οι λειτουργίες διαχειριστή προστέθηκαν ένα την φορά και δοκιμάστηκαν απευθείας. Όποτε εμφανιζόταν κάποιο θέμα κατά την ανάπτυξη, αυτό διορθωνόταν απευθείας στον ίδιο κύκλο χωρίς να επηρεάζει το συνολικό πλάνο. Αυτή η προσέγγιση βοήθησε ώστε να διατηρείται μία λειτουργική έκδοση της πλατφόρμας συνεχώς.
\\
\\
Το backend ακολούθησε την ίδια επαναληπτική λογική. Κάθε τελικό σημείο στο API αναπτύχθηκε ως ένα ανεξάρτητο κομμάτι και δοκιμάστηκε με την εφαρμογή και ιστοσελίδα στον ίδιο κύκλο. Αυτό κρατούσε την όλη διαδικασία οργανωμένη και διευκόλυνε την διόρθωση των σφαλμάτων, καθώς οποιοδήποτε πρόβλημα εμφανιζόταν από νωρίς.

\subsection{Waterfall}
Το σχέδιο της βάσης δεδομένων ακολούθησε μία διαφορετική στρατηγική, αυτή του Καταρράκτη (Waterfall), η οποία ταίριαζε καλύτερα στις ανάγκες της βάσης. Η μεθοδολογία Καταρράκτη ακολουθεί μία γραμμική ροή ανάπτυξης. Η λογική της είναι η ύπαρξη ενός ξεκάθαρου σχεδίου και η σταδιακή πλήρης υλοποίηση του. Σε αντίθεση με την Agile, αυτή δίνει έμφαση στην σταθερότητα και στον ορισμό της δομής πριν την ανάπτυξη.

\begin{figure}[H]
\centering
\includegraphics[width=0.55\textwidth]{images/waterfall.png}
\caption{Στρατηγική Προσέγγισης Waterfall}
\label{fig:waterfall}
\end{figure}

Αφού, η εφαρμογή και η ιστοσελίδα εξαρτώνται από την ίδια δομή δεδομένων, ήταν σημαντικό για τις συλλογές της βάσης και τα πεδία τους να είναι ξεκάθαρα ορισμένα από την αρχή. Οι δύο κύριες συλλογές, “users” και “reports”, είχαν καθοριστεί από το αρχικό στάδιο και παρέμειναν σταθερές σε όλη την διάρκεια της ανάπτυξης. 
\\
\\
Η σταθερή αυτή δομή συνέβαλε στην αποφυγή προβλημάτων συμβατότητας και επέτρεψε στο υπόλοιπο σύστημα να εξελιχθεί χωρίς συνεχείς αλλαγές στο μοντέλο δεδομένων. Η προσέγγιση Καταρράκτη ήταν κατάλληλη για αυτή την περίπτωση, επειδή η ιδέα της βασίζεται στην ολοκλήρωση κάθε φάσης χωρίς να απαιτείται να επανασχεδιαστεί ο αρχικός πυρήνας.