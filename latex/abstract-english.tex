In this thesis, a smart city reporting platform is designed and implemented to connect the citizens with municipality services so they can manage everyday urban problems more easily and effectively. The purpose of it is to replace slow and manual methods, such as phone calls and visits in person, with a simpler approach. The idea is that residents take a photo of the issue and report it to the system. Afterwards, the authorities handle those reports, which are displayed in an interactive map so that both sides can track their progress. The platform uses a backend created with Node.js and Express, a cloud database made with MongoDB Atlas, the Ionic React framework for the mobile application and the Leaflet library for the map display on the website. This project covers the requirements, architecture and implementation of such a system. From a design perspective, the platform follows a three tier approach, separating each part and keeping them all connected through a structured API to help in maintaining and making the development easier. The final product was evaluated using a questionnaire with 27 participants, who provided answers on a Likert scale ranging from one to five. The results showed high scores and an overall positive experience, with the questioners showcasing an intention to continue using the app and that it manages to achieve its goal of uniting the citizens with the municipalities.