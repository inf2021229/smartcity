\section{Έξυπνη Πόλη}

\drop{O}{όρος} “έξυπνη πόλη” έχει γίνει αρκετά διάσημος τα τελευταία χρόνια. Πολλές πόλεις γύρω στον κόσμο προσπαθούν να χρησιμοποιήσουν στρατηγικές και τεχνολογίες που θα κάνουν το περιβάλλον τους πιο αποδοτικό, φιλικό και βιώσιμο για τους πολίτες τους. Δεν υπάρχει όμως μια σταθερή έννοια για την έξυπνη πόλη. Υπάρχουν πολλές εκδοχές ως προς τι σημαίνει πραγματικά. Αυτό γίνεται επειδή, η ιδέα της έξυπνης πόλης συνδέεται με πολλές και διαφορετικές περιοχές όπως τις τεχνολογίες πληροφορικής και επικοινωνίας (ΤΠΕ), την οικονομική βελτίωση της πόλης και την διακυβέρνηση της ίδιας. Σύμφωνα με τους \textcite{macadar2016smart}, η έννοια της άρχισε να γίνεται παγκοσμίως γνωστή κατά τις αρχές του 2000 και έχει από τότε ανεπτυχθεί πλέον να περιέχει νέα στοιχεία όπως την άμεση συμμετοχή των πολιτών της, υποδομές στο ψηφιακό κομμάτι και διάφορες υπηρεσίες χρησιμοποιώντας δεδομένα.
\\
\\
Σε γενικό κομμάτι, μια έξυπνη πόλη μπορεί να προσδιοριστεί ως μια πόλη που χρησιμοποιεί δεδομένα και τεχνολογίες ώστε να βελτιώσει την εμπειρία της ζωής των πολιτών της και να κάνει τις υπηρεσίες των δήμων πιο αποτελεσματικές. Επιπλέον, εξίσου σημαντικό, να καταφέρει να περιορίσει όσο το δυνατό περισσότερο τις περιβαλλοντικές επιπτώσεις της σημερινής κοινωνίας \parencite{macadar2016smart}. Άρα, ως ορισμός η έξυπνη πόλη δεν έχει να κάνει απλά με την εφαρμογή τεχνολογιών και την υλοποίηση κινητών εφαρμογών αλλά έχει να κάνει με την δημιουργία ενός σωστού οικοσυστήματος όπου συλλέγονται πληροφορίες εύκολα και μπορούν αυτές να χρησιμοποιηθούν για την επίλυση των διάφορων προβλημάτων όπως η ρύπανση, η καλύτερη απόδοση της ενέργειας και διαχείριση διάφορων καταστροφών.

\subsection{Χαρακτηριστικά Έξυπνης Πόλης}
Οι έξυπνες πόλεις μεταξύ τους συνήθως μπορεί να ακολουθούν και να μοιράζονται τα ίδια χαρακτηριστικά. Συχνά, χρησιμοποιούν προηγμένα συστήματα ΤΠΕ ώστε να συλλέξουν και να επεξεργαστούν στοιχεία, τα οποία στην συνέχεια βοηθούν στην λήψη περαιτέρω αποφάσεων. Για παράδειγμα, ένας κυκλοφοριακός αισθητήρας μπορεί να παρατηρήσει συμφόρηση στους δρόμους και να προτείνει εναλλακτικές κατευθύνσεις προς τους οδηγούς σε πραγματικό χρόνο. Σύμφωνα με τους \textcite{arroub2016literature}, αυτά τα συστήματα συνηθίζουν να βασίζονται πάνω σε τεχνολογίες του διαδικτύου των πραγμάτων (IoT) και αναλύσεις μεγάλων δεδομένων ώστε να μπορούν να λειτουργήσουν σωστά και αποδοτικά.
\\
\\
Ένα από τα πιο σημαντικά χαρακτηριστικά, είναι η συμμετοχή των πολιτών. Μια πόλη δεν μπορεί απλά να θεωρηθεί “έξυπνη” μόνο με βάση τις τεχνολογίες που έχει υλοποιήσει χωρίς να περιλαμβάνει τους ίδιους τους ανθρώπους της στην όλη διαδικασία. Οι πολίτες έχουν τον δικό τους ρόλο, με την χρήση εφαρμογών μπορούν να αναφέρουν πιθανά προβλήματα, να πάρουν πρόσβαση στις δημόσιες υπηρεσίες της πόλης και να δώσουν ανατροφοδότηση πίσω σε αυτή. Με αυτή την λογική, η δημιουργία διάφορων εφαρμογών είναι πολύ σημαντική καθώς ενώνουν τον δήμο με την κοινωνία της πόλης \parencite{arroub2016literature}.
\\
\\
Επιπρόσθετα, οι έξυπνες πόλεις έχουν ως βασικό στόχο την βιωσιμότητα. Όπως εξηγεί ο \textcite{anthopoulos2015defining}, η σχεδίαση μια τέτοιας πόλης είναι άμεσα συνδεδεμένη με την ανάπτυξη της περιβαλλοντικά. Άρα, σκοπός της είναι η βελτίωση κατανάλωσης ενέργειας και η μείωση εκπομπών αερίων στην ατμόσφαιρα. Αυτή η σύνδεση είναι πολύ σημαντική σε αυτό το κομμάτι επειδή οι κατοικημένες περιοχές καταναλώνουν τεράστιας ποσότητας ενέργεια και παράγουν πολλή ποσότητα απόβλητων.

\subsection{Θετικά Έξυπνης Πόλης}
Υπάρχουν πολλά θετικά στην υλοποίηση έξυπνων πόλεων και ένα από τα πιο βασικά είναι πως βελτιώνουν την εμπειρία ζωής των πολιτών τους. Οι ίδιοι μπορούν να απολαύσουν τα διάφορα πλεονεκτήματα της όπως τα βελτιωμένα μέσα μεταφοράς και η ευκολότερη πρόσβαση σε υπηρεσίες της. Με βάση τους \textcite{arroub2016literature}, αυτές οι βελτιώσεις κάνουν τις ίδιες τις πόλεις πιο φιλικές προς τους τουρίστες και τις εταιρείες κάτι το οποίο βοηθάει στην ανάπτυξη της οικονομίας τους.
\\
\\
Ακόμα, η χρήση τεχνολογιών για την παρακολούθηση και την ελαχιστοποίηση κατανάλωσης, και την αποδοτική διαχείριση ρύπων οδηγεί σε σημαντικά οφέλη. Οι \textcite{drakouli2019} δίνουν έμφαση πως μέσα στην ΕΕ, οι έξυπνες πόλεις θεωρούνται ως βασική στρατηγική για την καταπολέμηση της κλιματικής αλλαγής και για την επίτευξη των υπόλοιπων τους στόχων για το περιβάλλον. Πιο συγκεκριμένα, έξυπνοι μετρητές μπορούν να βοηθήσουν ένα καταναλωτή να επιβλέπει πόση ενέργεια ξοδεύει και να διαμορφώσουν καλύτερα την διανομή ηλεκτρικής ενέργειας.
\\
\\
Τέλος, μέσω αναλύσεων και αυτοματοποιήσεων, οι αρχές του δήμου μπορούν να δουλέψουν πιο αποτελεσματικά, να πάρουν καλύτερες αποφάσεις και να αντιδρούν ταχύτερα σε επείγοντα θέματα. Ο \textcite{anthopoulos2015defining} παρατηρεί πως αυτή η αποτελεσματικότητα προέρχεται από την ενσωμάτωση διαφορετικών συστημάτων και την κοινοποίηση πληροφοριών μεταξύ των τμημάτων του.

\subsection{Προκλήσεις και Περιορισμοί}
Οι έξυπνες πόλεις παρά τα πολλά πλεονεκτήματα τους, αντιμετωπίζουν πολλές προκλήσεις. Ένα από τα πιο σημαντικά μειονεκτήματα τους είναι το κόστος της υλοποίησης μιας τέτοιας ιδέας. Το να κτίζεις υποδομές και να αναπτύσσεις εφαρμογές λογισμικού χρειάζεται πολλές επενδύσεις. Αρκετές πόλεις και ειδικά αυτές που είναι σε χώρες που αναπτύσσονται ακόμα δεν μπορούν να φτάσουν αυτά τα ποσά στα έξοδα τους. Ο \textcite{konbr2019smart} δίνει ένα τέτοιο παράδειγμα με την πόλη της Αιγύπτου, όπου τα πλάνα τους για μια τέτοια πόλη είναι ο περιορισμός κατανάλωσης ενέργειας και καλύτερη διαχείριση αποβλήτων. Ωστόσο, στο άρθρο του επίσης δείχνει πως αυτά τα πλάνα είναι αρκετά δύσκολα για να γίνουν πραγματικότητα λόγω οικονομικών προκλήσεων και οργάνωσης. 
\\
\\
Άλλο ένα σημαντικό πρόβλημα είναι η ασφάλεια των δεδομένων. Αφού η συλλογή προσωπικών πληροφοριών παίζει τόσο σημαντικό ρόλο, υπάρχει πάντα το ρίσκο κυβερνοεπίθεσης και κακόβουλης χρήσης αυτών. Όπως υποστηρίζει ο \textcite{anthopoulos2015defining}, η ασφάλεια θα πρέπει να ενσωματωθεί από την αρχή μέσα στην αρχιτεκτονική των συστημάτων, με την χρήση κρυπτογράφησης για τα δεδομένα, ασφαλισμένων πρωτόκολλων επικοινωνίας και συνεχή έλεγχου.

Υπάρχει ακόμα ο περιορισμός ως προς την τεχνολογία. Αρκετοί πολίτες μπορεί να μην έχουν το ίδιο επίπεδο πρόσβασης σε τεχνολογία κάτι το οποίο δημιουργεί ανισότητα. Οι \textcite{drakouli2019} τονίζουν πως οι στρατηγικές τέτοιων περιοχών θα πρέπει να αποφύγουν τον διαχωρισμό των πολιτών σε διαφορετικές ψηφιακές ομάδες.
\\
\\
Τέλος, η ρύθμιση μιας έξυπνης πόλης μπορεί να είναι περίπλοκη. Για την εκτέλεση τέτοιων λύσεων συχνά είναι υποχρεωτική η συνεργασία μεταξύ πολλαπλών πλευρών όπως κυβερνήσεων, εταιρειών και πολιτών αντίστοιχα. Χωρίς την σωστή διαχείριση και ξεκάθαρων υποχρεώσεων, σχέδια σαν αυτά ενδέχονται να αποτύχουν και να εμφανίσουν συγκρούσεις \parencite{anthopoulos2015defining}.

\subsection{Μέλλον των Έξυπνων Πόλεων}
Κοιτάζοντας στο μέλλον, οι έξυπνες πόλεις αναμένονται με την συνεχή ανάπτυξη σε όλους τους τομείς να εμφανίζονται ακόμα πιο συχνά γύρω στον κόσμο. Με την χρήση τεχνητής νοημοσύνης, ανανεωμένων δικτύων και αναλύσεων μεγάλων δεδομένων, οι πόλεις θα μπορούν να παρέχουν πιο προχωρημένες υπηρεσίες \parencite{macadar2016smart}. Παρόλα αυτά, η επιτυχία τους προχωρώντας μπροστά εξαρτάται σημαντικά από την αντιμετώπιση τωρινών προκλήσεων όπως η ασφάλεια και το κόστος. Αυτά τα θέματα θα πρέπει να επιλυθούν ώστε να μπορούν να θεωρηθούν δίκαιες αλλά και βιώσιμες για όλους, καθώς δεν είναι αρκετή μόνο η επικέντρωση στην τεχνολογία \parencite{arroub2016literature}.
\\
\section{Κινητές και Διαδικτυακές Εφαρμογές}
Οι τωρινές έξυπνες πόλεις δεν μπορούν να λειτουργήσουν αποτελεσματικά χωρίς να υπάρχει επικοινωνία μεταξύ των πολιτών και του δήμου. Ενώ μέσα στην έννοια εμπλέκονται αρκετά οι ανεπτυγμένες υποδομές και τα IoT συστήματα, η πραγματική αξία έρχεται όταν όλα αυτά είναι διαθέσιμα με ευκολία στους ανθρώπους όπου ζουν εκεί. Οι κινητές και διαδικτυακές εφαρμογές αποτελούν τα απαραίτητα εργαλεία ώστε να γίνει κάτι σαν αυτό πραγματικότητα. Αυτές οι εφαρμογές επιτρέπουν τους κατοίκους να μπορούν να αλληλεπιδράσουν με τις τοπικές αρχές, να αναφέρουν διάφορα ζητήματα και το πιο σημαντικό ακόμα να έχουν παρόν στην λήψη αποφάσεων. Όπως εξηγούν οι \textcite{hou2020road} , η επιτυχία αυτών των αστικών κυβερνύσεων συχνά βασίζονται στο πόσο καλά οι “κινητές” λύσεις έχουν ενσωματωθεί στις υπηρεσίες τους. 

Η άνοδος των τηλεφώνων και η διάδοση της πρόσβασης στο ίντερνετ έχουν δημιουργήσει ένα νέο περιβάλλον στο οποίο όλες οι υπηρεσίες μπορούν πλέον να γίνονται ψηφιακά. Οι άνθρωποι δεν χρειάζονται πλέον να έρθουν σε επαφή με δημοτικά γραφεία για κάθε αίτημα τους. Αντίθετα, μπορούν να χρησιμοποιήσουν εφαρμογές για την υποβολή παραπόνων και δημόσιων έργων καθώς και να ενημερώνονται σε πραγματικό χρόνο πάνω σε αυτά. Αυτή η αλλαγή σε ηλεκτρονικές λειτουργίες αποτελεί μια από τις βασικές πλευρές των έξυπνων πόλεων της σημερινής εποχής.

\subsection{Ο ρόλος των Κινητών Εφαρμογών}
Ένα από τα πιο συχνά εμφανιζόμενα στοιχεία των έξυπνων πόλεων, ως προς τις λύσεις για έναν μέσο κάτοικο, αποτελούν οι κινητές εφαρμογές. Σύμφωνα με τους \textcite{hou2020road}, αυτές οι εφαρμογές δεν είναι απλά άλλο ένα κανάλι επικοινωνίας αλλά επηρεάζουν ριζικά το πως δουλεύει η διακυβέρνηση. Για παράδειγμα, η χρήση τους πάνω στην διαχείριση της πόλης επιτρέπει τον πολίτη να αναφέρει λακκούβες και σπασμένα φώτα δρόμου, βγάζοντας απλά μόνο μια φωτογραφία και στέλνοντας την μέσω από αυτές. Αυτό δημιουργεί ένα άμεσο τρόπο ανατροφοδότησης μεταξύ της κοινωνίας και των αρχών μειώνοντας τους χρόνους απάντησης και βελτιώνοντας την υπευθυνότητα.
\\
\\
Οι \textcite{boulos2015social} δίνουν έμφαση πως τέτοιες εφαρμογές παίζουν τεράστιο ρόλο στην καινοτομία και στην συμμετοχή της κοινότητας. Στον τομέα της υγείας, για παράδειγμα, μπορεί να τους δωθεί η δυνατότητα να παρακολουθούν και να εντοπίζουν πιθανές μεταβολές στα επίπεδα ρύπανσης. Παρομοίως, εφαρμογές που έχουν φτιαχτεί τελείως για συμμετοχή της κοινότητας μπορούν να αφήσουν τους πολίτες να ψηφίζουν σε τοπικά έργα και να δίνουν τις προσωπικές τους αντιδράσεις σε διάφορες αστικές προτάσεις. Αυτές οι λειτουργίες δίνουν στις κινητές εφαρμογές ένα ρόλο πιο σημαντικό από απλά μέσα για την διευκόλυνση, κάνοντας τα εργαλεία της συμμετοχικής διαχείρισης των πόλεων.

\subsection{Εφαρμογές Ιστού ως Συμπληρωματικά Εργαλεία}
Οι κινητές εφαρμογές είναι απαραίτητες για την εν κινήσει πρόσβαση, όμως στο κομμάτι της προσέγγισης οι διαδικτυακές εφαρμογές παραμένουν πιο σημαντικές. Συνήθως, οι πολίτες προτιμούν την χρήση διεπαφών στον υπολογιστή για πιο σημαντικές εργασίες όπου χρειάζονται πιο περίπλοκες ενέργειες. Με βάση τους \textcite{aslam2020comprehensive}, οι εφαρμογές ιστού είναι συχνά ενωμένες με την ίδια υποδομή backend με αυτή στις κινητές, διασφαλίζοντας σταθερότητα και σύνδεση σε πραγματικό χρόνο μεταξύ τους. Άρα, ένα αίτημα από μια εφαρμογή κινητού μπορεί να εμφανίζεται στην διαδικτυακή σελίδα μιας πόλης, δίνοντας στους χρήστες πολλαπλές πλατφόρμες για να αλληλεπιδράσουν. Ακόμα, διαδικτυακοί πίνακες ελέγχου σε σελίδες μπορούν να βοηθήσουν τους διαχειριστές στην οπτικοποίηση των συλλεγμένων δεδομένων από τους κατοίκους. Αυτοί οι πίνακες, ενσωματώνονται σε συστήματα IoT και δείχνουν πληροφορίες από σένσορες, κάμερες και αναφορές σε πραγματικό χρόνο. Αυτός ο συνδυασμός των κινητών και διαδικτυακών πλατφορμών είναι καθοριστικός στο να κάνει τα οικοσυστήματα να δουλεύουν ομαλά.

\subsection{Μελέτες Περιπτώσεων και Εφαρμογές}
Στην Ελλάδα, οι ψηφιακές στρατηγικές των δήμων έχουν άρχισει να εισάγουν κινητές και διαδικτυακές πλατφόρμες ως πρωτοβουλίες για τις έξυπνες πόλεις. Οι \textcite{siokas2019} επισημαίνουν πως πολλές ελληνικές πόλεις τις έχουν υιοθετήσει ήδη για απλές υπηρεσίες όπως την ενημέρωση του πολίτη και την ηλεκτρονική διαχείριση της κυβέρνησης, παρόλο που είναι ακόμα υπό ανάπτυξη. Ωστόσο, η έρευνα τους σημειώνει και προκλήσεις όπου μικρότερες πόλεις δυσκολεύονται να υιοθετήσουν τέτοια συστήματα και υπάρχει έλλειψη τεχνικής υποστήριξης σε αυτές.
\\
\\
Στο εξωτερικό, οι \textcite{angelopoulos2019} μελέτησαν πραγματικές χρήσεις αυτών των εφαρμογών, δείχνοντας το πως οι ψηφιακές λύσεις είναι ενσωματωμένες στις δημοτικές υπηρεσίες. Η ανάλυση τους έδωσε σημασία στην εμπιστοσύνη των πολιτών και την ευκολία χρήσης ως κύριο κλειδί για την έγκριση τους. Παρομοίως, οι \textcite{hou2020road} έκαναν μια μελέτη πάνω σε εφαρμογές διαχείρισης της πόλης και βρήκαν πως η αποδοχή από τους χρήστες εξαρτάται από την γρήγορη ενημέρωση και την αξιοπιστία. Οι πολίτες, επομένως, είναι πιο πρόθυμοι να χρησιμοποιήσουν αυτές τις πλατφόρμες όταν παρατηρούν συνεχή ανταπόκριση από τις αρχές. Αντίστοιχα, αν ζητήματα παραμένουν άλυτα χωρίς καθόλου ενημερώσεις, τότε η εμπιστοσύνη τους σε αυτά μειώνεται. Αυτό δείχνει πως οι δήμοι θα πρέπει να βελτιωθούν εσωτερικά ως προς την ροή εργασίας ώστε να μπορούν να εγγυηθούν γρήγορες υπηρεσίες στο κοινό τους. 

\subsection{Ενσωμάτωση με IoT και άλλων τεχνολογιών}
Οι εφαρμογές αυτές συνήθως θεωρούνται ως διεπαφές χρηστών για πιο περίπλοκα IoT συστήματα. Σύμφωνα με τους \textcite{aslam2020comprehensive}, πολλές υπηρεσίες των πόλεων βασίζονται πάνω σε δεδομένα όπου συλλέγονται από αισθητήρες που έχουν τοποθετηθεί γύρω σε αυτές. Κάποια είδη αυτών είναι οι σένσορες για την ποιότητα αέρα όπου μπορούν να μεταφέρουν πληροφορίες στους κατοίκους, ειδοποιώντας τους για τα επίπεδα μόλυνσης. Ακόμα, μπορούν να βοηθήσουν και σε έξυπνα συστήματα πάρκινγκ όπου δείχνουν τις διαθέσιμες θέσεις των χώρων αυτών σε πραγματικό χρόνο. Η προσθήκη αυτών σε συνεργασία με τις εφαρμογές κάνουν άμεσα διαθέσιμες σημαντικές πληροφορίες για τους ανθρώπους.
\\
\\
Το backend κομμάτι αυτών των υποδομών συχνά αποτελείται από υπηρεσίες στο cloud όπου αποθηκεύουν και επεξεργάζονται τα δεδομένα. Ακόμα, χρησιμοποιούνται APIs όπου συνδέουν τις εφαρμογές με τις βάσεις δεδομένων και εργαλεία ασφαλείας για την προστασία απορρήτων πληροφοριών. Οι \textcite{angelopoulos2019} δίνουν έμφαση στο πόσο απαραίτητη είναι εμπιστοσύνη σε αυτά τα οικοσυστήματα. Οι χρήστες θα μοιραστούν τα προσωπικά τους δεδομένα μόνο αν πιστέψουν πως αυτά θα προστατευτούν από κακόβουλες χρήσεις. Επομένως, είναι αναγκαίο να υπάρχει δυνατή κρυπτογράφηση και ξεκάθαρες πολιτικές προστασίας ως προς την χειραγώγηση των δεδομένων ώστε οι πλατφόρμες αυτές να θεωρηθούν επιτυχής.

\subsection{Οφέλη και Εμπόδια}
Η χρήση των εφαρμογών στις έξυπνες πόλεις εμφανίζει πολλά οφέλη. Οι κάτοικοι έχουν ευκολότερη πρόσβαση σε υπηρεσίες, μπορούν να αναφέρουν οποιαδήποτε προβλήματα στην στιγμή και να παραλάβουν νέα χωρίς την φυσική τους παρουσία. Αυτό βελτιώνει την σχέση των χρηστών με τις τοπικές αρχές εφόσον έχουν παραπάνω έλεγχο καθώς ενισχύεται η ευκολία χρήσης και η διαφάνεια μεταξύ τους \parencite{hou2020road}. Επιπλέον, δημιουργούνται νέες ευκαιρίες για ενεργή συμμετοχή στην λήψη αποφάσεων βοηθώντας τους πολίτες να νιώθουν πως εμπλέκονται παραπάνω στις κοινότητές τους \parencite{boulos2015social}. Για τους δήμους, η συνεχή ροή δεδομένων από τους ανθρώπους μαζί με τις IoT συσκεύες τους βοηθάει στην καλύτερη κρίση επιλογών και πιο αποτελεσματική χρήση πόρων.
\\
\\
Όμως, υπάρχουν πολλά εμπόδια για να εφαρμοστεί αυτό επιτυχημένα. Οι \textcite{siokas2019} αναφέρουν πως πολλοί δήμοι έρχονται αντιμέτωποι με οικονομικούς περιορισμούς και έλλειψη τεχνικής εμπειρίας, κάτι το οποίο επιβραδύνει στην ανάπτυξη των κινητών υπηρεσιών. Ακόμα, η ανησυχία για την προστασία δεδομένων και ασφάλειας μπορούν να περιορίσουν τον συνολικό αριθμό χρηστών, όπως σημειώνουν οι \textcite{angelopoulos2019}. Άλλο ένα θέμα είναι η ανισότητα στον ψηφιακό χώρο. Αυτά τα εμπόδια πρέπει να αντιμετωπιστούν μέσω της εκπαίδευσης των πολιτών και δυνατών στρατηγικών ώστε να διασφαλίσουν πως όλοι θα κερδίσουν από τις πρωτοβουλίες αυτές.
\\
\section{GIS}
Τα γεωγραφικά συστήματα πληροφοριών (GIS) είναι απαραίτητα για την διαχείριση και ανάλυση δεδομένων στις έξυπνες πόλεις. Αυτά δίνουν την δυνατότητα στους δήμους και στους πολίτες να οπτικοποιήσουν σύνθετα σετ στοιχείων και να πάρουν αποφάσεις με βάση τις πληροφορίες βασισμένες στην τοποθεσία. Σε μια έξυπνη πόλη, ένα μεγάλο μέγεθος γεγονότων μπορεί να συνδεθεί άμεσα και με μια γεωγραφική θέση. Τα GIS βοηθάνε στην οργάνωση και στην επίδειξη αυτών με τρόπο εύκολο ως προς την κατανόηση και στην επεξεργασία τους \parencite{zhaoexploring}. 
\\
\\
Τέτοια διαδικτυακά εργαλεία έχουν άρχισει να γίνονται πιο σημαντικά επειδή αφήνουν οποιονδήποτε που έχει πρόσβαση στο ίντερνετ να μπορεί να βλέπει ελεύθερα αυτά τα χωρικά δεδομένα. Σε αντίθεση με τα παραδοσιακά GIS λογισμικά, όπου μπορεί να είναι ακριβά και χρειάζονται ειδικά μηχανήματα, τα τωρινά εργαλεία χρησιμοποιούν πλαίσια ανοικτού κώδικα για την προβολή χαρτών και δεδομένων σε απλούς ιστότοπους.

\subsection{Ο ρόλος των GIS στις Έξυπνες Πόλεις}
Τα γεωγραφικά συστήματα παίζουν ένα σημαντικό ρόλο στην ανάπτυξη μια έξυπνης πόλης. Οι \textcite{zhaoexploring} σημειώνουν πως οι αναλύσεις από αυτά τα εργαλεία παρέχουν γεωγραφικά πλαίσια όπου χρειάζονται για την αποτελεσματική διαχείριση πόρων. Για παράδειγμα, μπορούν να αναγνωρίσουν μοτίβα κατανάλωσης σε διαφορετικές γειτονιές και να εντοπίσουν κυκλοφοριακές συμφορήσεις. Αυτές οι ιδέες είναι απαραίτητες για βιωσιμότητα και λειτουργική αποτελεσματικότητα. 
\\
\\
Σύμφωνα με τους \textcite{bovkir2021big}, αυτά τα συστήματα ενσωματώνουν δεδομένα πραγματικού χρόνου από αισθητήρες IoT. Με τον συνδυασμό αυτών, οι πλατφόρμες δίνουν την δυνατότητα στις πόλεις να αντιδρούν γρήγορα σε αλλαγές. Οπότε, αν σε μια περιοχή οι ελεγκτές αέρα παρατηρήσουν κάποια μεγάλη τιμή μόλυνσης, οι αρχές μπορούν να εκδώσουν προειδοποιήσεις άμεσα και να δράσουν αντίστοιχα. Αυτή η διαδικασία αποφάσεων σε ταχύ χρόνο δεν θα ήταν δυνατή χωρίς την ανάλυση χωρικών πληροφοριών που προσφέρουν τα GIS. 
\\
\\
Επιπρόσθετα, τα GIS μπορούν να υποστηρίξουν προγνωστικές λειτουργίες, μοντελοποιώντας μελλοντικά σενάρια με βάση παλαιών και τωρινών στοιχείων. Η συγκεκριμένη δυνατότητα είναι ιδιαίτερα χρήσιμη κατά την σχεδίαση όπου οι υπεύθυνοι μπορούν να προσομοιώσουν την επιρροή νέων αποφάσεων. Όλα αυτά κάνουν τα GIS μια βασική τεχνολογία στις στρατηγικές των έξυπνων πόλεων

\subsection{Συμμετοχικό GIS}
Μία σημαντική τάση στα γεωγραφικά συστήματα των έξυπνων πόλεων είναι η εισαγωγή συμμετοχικών προσεγγίσεων όπου οι κάτοικοι συνεισφέρουν δικά τους χωρικά δεδομένα μέσω την χρήση εφαρμογών. Οι \textcite{bakowska2021use} παρουσιάζουν μια αξιολόγηση των GIS με δημόσια συμμετοχή (PPGIS) και δίνουν έμφαση στον ρόλο τους στην στήριξη των τοπικών κοινοτήτων ώστε να συμμετάσχουν στην αστική σχεδίαση μέσω διαδικτυακών περιβαλλόντων χαρτογράφησης. Εξηγούν πως το PPGIS ενισχύει την επικοινωνία και την ενδυνάμωση των πολιτών δίνοντας την ευκαιρία σε αυτούς να εμπλακούν άμεσα στην συλλογή γεωγραφικών δεδομένων και στην παραγωγή αποφάσεων. Όταν συνεργάζονται με εργαλεία σαν το Leaflet, τα PPGIS μπορούν να διευκολύνουν την δημιουργία φιλικών προς τον χρήστη πλατφορμών που προωθούν την γενική συμμετοχή του κοινού σε σημαντικές πρωτοβουλίες των πόλεων. 
\\
\section{Εργαλεία Ιστοσελίδας}
Για την διαδικτυακή απεικόνιση των δεδομένων, η παρούσα εργασία υιοθετεί μία ελαφριά στοίβα HTML και JavaScript με διαδραστικό χάρτη μέσω του Leaflet. Το Leaflet επιλέγεται διότι συνδυάζει απλότητα, επεκτασιμότητα και άριστη προσαρμογή σε κινητά περιβάλλοντα, χαρακτηριστικά που ταιριάζουν με τις ανάγκες ενός δημοτικού συστήματος αναφορών. Ο ίδιος ο ιστότοπος επικοινωνεί με το backend αποκλειστικά μέσω REST API, ώστε η απεικόνιση και η διαχείριση των αναφορών να είναι ανεξάρτητες από την υλοποίηση του διακομιστή.

\subsection{Leaflet}
Το Leaflet είναι μία από τις πιο ευρέως χρησιμοποιημένες βιβλιοθήκες για την δημιουργία γεωγραφικών εφαρμογών. Σύμφωνα με την επίσημη τεκμηρίωση του \parencite{Leaflet}, το Leaflet είναι μια ανοικτού κώδικα βιβλιοθήκη της JavaScript σχεδιασμένη για την εύκολη δημιουργία εφαρμογών με χάρτες. Είναι αρκετά ελαφριά, γρήγορη και υποστηρίζει βασικά χαρακτηριστικά για χάρτες όπως ζουμ, δείκτες και η προβολή αναδυόμενων παραθύρων. Η απλοϊκότητα του και το εύρος δυνατοτήτων του το κάνει μία αγαπημένη επιλογή για προγραμματιστές που θέλουν να εισχωρήσουν χάρτες μέσα στις εφαρμογές του χωρίς πολύπλοκα GIS λογισμικά. 
\\
\\
Στο πλαίσιο των έξυπνων πόλεων, η ικανότητα του Leaflet να εισάγει εξωτερικά APIs και άλλα προγράμματα το κάνει ένα αρκετά χρήσιμο εργαλείο. Ακόμα, το Leaflet υποστηρίζει διάφορες μορφές γεωγραφικών δεδομένων οπότε μπορεί εύκολα να διαχειριστεί χωρικά σετ πληροφοριών από IoT συσκευές και βάσεις δεδομένων των πόλεων. Ένα παράδειγμα χρήσης του είναι, να μπορούν οι προγραμματιστές να εμφανίσουν τοποθεσίες δημόσιων υποδομών ή και να σημαδέψουν προβλήματα που έχουν καταχωρήσει οι πολίτες.
\\
\\
Οι \textcite{karampakakis2025web} δίνουν μια περίπτωση για το πως χρησιμοποιείται το Leaflet σε πράξη στις έξυπνες πόλεις. Στην μελέτη τους, ανέπτυξαν μια διαδικτυακή πλατφόρμα για την ανάλυση και προβολή δεδομένων ανίχνευσης σε αστικές περιοχές. Η εφαρμογή συνδύαζε γραφήματα χρόνου με αλληλεπιδράσιμους χάρτες, επιτρέποντας στους χρήστες να παρατηρήσουν περιβαλλοντικά στοιχεία όπως η θερμοκρασία και η ποιότητα αέρα σε διαφορετικά μέρη της πόλης. Το Leaflet είχε ρόλο του βασικού στοιχείου χαρτογράφησης, παρέχοντας δυναμική οπτικοποίηση και ικανότητες διαδραστικότητας για τους χρήστες.
\\
\\
Πέρα από το Leaflet, υπάρχουν πολλά πλαίσια ανοικτού κώδικα που υποστηρίζουν την ανάπτυξη GIS πλατφορμών Οι \textcite{sejati2020open} συζητάνε για πλαίσια όπου χρησιμοποιούν ανοικτά πρότυπα και τεχνολογίες. Η δικιά τους προσέγγιση περιλαμβάνει την εισαγωγή βάσεων με γεωγραφικά δεδομένα με εργαλεία για οπτικοποίηση μέσα από το διαδίκτυο, δίνοντας την δυνατότητα σε χρήστες να παρακολουθούν τυχόν αλλαγές διαχρονικά. Ενώ το Leaflet μπορεί να διαχειριστεί μόνο του το οπτικό κομμάτι, τέτοια πλαίσια παρέχουν το λειτουργικό μέρος της υποδομής για την αποθήκευση και διαχείριση των πληροφοριών. Η χρήση τέτοιων εργαλείων είναι εξίσου σημαντική για μικρότερες πόλεις που δεν μπορούν να διαθέσουν ακριβά GIS λογισμικά. Με την χρήση αυτών των πλαισίων που συνδυάζουν το Leaflet με άλλες τεχνολογίες, οι πόλεις αυτές μπορούν να εφαρμόσουν προηγμένες χαρτογραφήσεις χωρίς μεγάλα έξοδα.
\\
\section{Τεχνολογίες Ανάπτυξης σε Κινητά}
Η ζήτηση για κινητές εφαρμογές αυξάνεται συνεχώς και ραγδαία με τις εταιρείες να πρέπει να αναπτύξουν εφαρμογές και για Android και για iOS πλατφόρμες. Η δημιουργία δύο διαφορετικών εφαρμογών μπορεί να θεωρηθεί χρονοβόρα και ακριβή. Για αυτό τον λόγο, έχουν γίνει διάσημα τα framework πολλαπλών πλατφορμών, τα οποία δίνουν την δυνατότητα σε έναν προγραμματιστή να γράψει σε μία βάση κώδικα όπου τρέχει σε πολλαπλές συσκευές μειώνοντας σε μεγάλο βαθμό το κόστος και τον χρόνο κατά την ανάπτυξη \parencite{majchrzak2017comprehensive}. Οι κύριες προσεγγίσεις σε αυτό τον τομέα είναι οι εγγενές, οι υβριδικές και οι προοδευτικές εφαρμογές ιστού (PWA). Οι εγγενές, αλλιώς native, είναι οι εφαρμογές όπου χρησιμοποιούν γλώσσες προγραμματισμού συγκεκριμένης πλατφόρμας όπως Java και Kotlin για Android και Swift για iOS. Αυτές προσφέρουν την καλύτερη απόδοση αλλά χρειάζονται ξεχωριστές βάσεις κώδικα για να λειτουργήσουν. Οι υβριδικές, στην άλλη πλευρά, χρησιμοποιούν συστήματα ιστού όπως HTML, CSS και JavaScript για την κατασκευή εφαρμογών όπου τρέχουν μέσα σε ένα native “δοχείο”. Με αυτή την προσέγγιση, συνδέεται η ελευθερία του προγραμματισμού στον ιστό με την πρόσβαση σε δυνατότητες των συσκευών μέσω native plugins. Τα PWAs προσφέρουν την εμπειρία του διαδικτύου υποστηρίζοντας χρήση χωρίς σύνδεση σε αυτό αλλά μειονεκτούν καθώς δεν έχουν πρόσβαση σε native APIs.
\\
\section{Εργαλεία Κινητής Εφαρμογής}
Στην παρούσα εργασία επιλέγεται η υβριδική προσέγγιση με Ionic React, TypeScript και Capacitor. Ο συνδυασμός αυτός επιτρέπει την χρήση μίας ενιαίας βάσης κώδικα και την αξιοποίηση του οικοσυστήματος της React, ενώ μέσω του Capacitor παρέχεται ασφαλή πρόσβαση σε λειτουργίες της συσκευής, όπως ο γεωεντοπισμός και η κάμερα. Η επιλογή ικανοποιεί τις λειτουργικές απαιτήσεις της πλατφόρμας (λήψη φωτογραφίας και αυτόματη καταγραφή τοποθεσίας) με χαμηλό κόστος ανάπτυξης και συντήρησης.

\subsection{Ionic Framework}
Το Ionic είναι ένα framework ανοικτού κώδικα όπου σχεδιάστηκε για την δημιουργία υβριδικών και πολλαπλών πλατφορμών εφαρμογών για κινητά χρησιμοποιώντας web τεχνολογίες. Όταν παρουσιάστηκε για πρώτη φορά το 2013, το Ionic ήταν αρχικά επικεντρωμένο σε Angular αλλά αργότερα επεκτάθηκε σε React και σε άλλα πλαίσια, κάνοντας το ιδανική επιλογή για προγραμματιστές \parencite{Ionicframework}. Το ίδιο προσφέρει μια βιβλιοθήκη γεμάτη με ήδη έτοιμα κομμάτια που βοηθούν στην δημιουργία UI, βελτιώνοντας απίστευτα την σχεδίαση εφαρμογών με ομοιόμορφη μορφή και λειτουργία σε όλες τις πλατφόρμες. 
\\
\\
Μία από τις πιο βασικές δυνάμεις του Ionic είναι η χρήση ενός μοναδικού κώδικα για τις πλατφόρμες κάτι το οποίο μειώνει την εξέλιξη του και τα κόστοι διατήρησης \parencite{Sonar2023Hybrid}. Αυτό είναι ειδικά χρήσιμο σε έργα όπως αυτό μιας έξυπνης πόλης όπου ο χρόνος κυκλοφορίας των εφαρμογών και η ελαχιστοποίηση δαπάνης των πόρων είναι κρίσιμα. Οι εφαρμογές του Ionic τρέχουν σε WebView όπου λειτουργεί σαν πρόγραμμα περιήγησης μέσα σε τοπικό περιβάλλον, δίνοντας τους την δυνατότητα να έχουν πρόσβαση σε λειτουργίες του κινητού ενώ διατηρεί την συμβατότητα ανάμεσα στις διάφορες συσκευές.
\\
\\
Ακόμα, είναι δυνατή η υποστήριξη PWAs. Έτσι, το Ionic μπορεί να εφαρμόσει την ίδια βάση κώδικα ως εφαρμογή διαδικτύου. Αυτή η ελευθερία σιγουρεύει πως οι εφαρμογές μπορούν να χρησιμοποιηθούν από οποιοδήποτε χρήστη σε κινητά, τάμπλετ καθώς και λάπτοπ \parencite{Ionicframework}. Για την μορφή της τελικής διεπαφής χρησιμοποιείται σύστημα με προσαρμόσιμα κομμάτια λογισμικού κάνοντας την εφαρμογή να δουλεύει σε οθόνες ανεξαρτήτως των διαστάσεων χωρίς παραπάνω δουλειά στον προγραμματισμό.

\subsubsection{React}
Η React είναι μια βιβλιοθήκη της JavaScript η οποία είναι ανεπτυγμένη από την Meta για την δημιουργία διεπαφών χρηστών. Η ίδια εισήγαγε την έννοια του εικονικού μοντέλου DOM, το οποίο βελτιώνει την απόδοση μειώνοντας τις άμεσες αλλαγές στο πραγματικό DOM \parencite{React}. Οι εφαρμογές μέσω React χτίζονται με την χρήση επαναχρησιμοποιήσιμων UI μπλοκ όπου μπορούν να συνδυαστούν και να κατασκευάσουν περίπλοκες διεπαφές. Επομένως, βελτιώνεται η οργάνωση του κώδικα και η συνεχής χρήση του κάτι που είναι ιδανικό ως επιλογή για πλαφόρμες που χρειάζονται επεκτασιμότητα.

\subsubsection{Ionic React}
Όπως αναφέρθηκε, το Ionic άρχισε βασισμένο στην Angular αλλά πλέον υποστηρίζει πολλαπλά framework για frontend συμπεριλαμβανομένης της React. Το Ionic React συνδυάζει τις λειτουργίες UI του Ionic με την αρχιτεκτονική της React, κάνοντας το ελκυστικό για όσους είναι οικείοι με το οικοσύστημα της. Με την χρήση των δύο μαζί εμφανίζονται πολλά πλεονεκτήματα. Αρχικά, το πιο σημαντικό είναι πως επιτρέπει στους προγραμματιστές να φτιάχνουν διεπαφές με έτοιμα εργαλεία, το οποίο βοηθάει στην συντήρηση του κώδικα. Ακόμα, χρησιμοιποιείται η γλώσσα TypeScript για την σύνταξη κώδικα, όπου εφαρμόζει ασφάλεια και μειώνει τα σφάλματα κατά την ώρα εκτέλεσης \parencite{TypeScript}. Αυτό παίζει μεγάλο ρόλο σε έργα μεγάλης διάστασης που η αξιοπιστία είναι καθοριστική.
\\
\\
Σε αυτή την εργασία, επιλέχθηκε το Ionic React επειδή προσφέρει και ευκολία στην ανάπτυξη αλλά και κατάλληλη απόδοση. Η χρήση της TypeScript ενισχύει περαιτέρω την επιλογή αφού βελτιώνει την ποιότητα του κώδικα κάνοντας την εφαρμογή πιο δομημένη και διατηρήσιμη.

\subsubsection{Capacitor}
Το Capacitor είναι ένα περιβάλλον εκτέλεσης ανοικτού κώδικα φτιαγμένο από την ομάδα του Ionic για την αντικατάσταση του Cordova. Χρησιμοποιείται ως ενδιάμεσο εργαλείο μεταξύ των τεχνολογιών ιστού με τις τοπικές λειτουργίες των συσκευών, επιτρέποντας την κλήση APIs από την JavaScript \parencite{Capacitor}. Το Capacitor υποστηρίζει πολλές υπηρεσίες όπως την πρόσβαση στην κάμερα, τοποθεσία και την αποθήκευση αρχείων οι οποίες είναι απαραίτητες για μοντέρνες εφαρμογές ειδικά στον τομέα της έξυπνης πόλης. Ένα από τα πολλά θετικά που έχει σε σχέση με παλιά εργαλεία όπως το Cordova είναι η καλύτερη ενσωμάτωση σε νέα frameworks. Έχει απλή διαμόρφωση και την δυνατότητα επαναφόρτωσης της διεπαφής σε πραγματικό χρόνο κατά την ώρα της ανάπτυξης, γεγονός που επιταχύνει την διαδικασία αυτή. Ακόμα, παρέχει χρήσιμα plugins όπου μπορούν να χρησιμοποιηθούν για διάφορες διαδικασίες και την δυνατότητα δημιουργίας καινούργιων από την αρχή.
\\
\\
Η επιλογή του Capacitor έγινε για την πρόσβαση στην κάμερα της συσκευής και για την εύρεση της γεωγραφικής τοποθεσίας. Αυτά τα χαρακτηριστικά είναι σημαντικά για την καταχώρηση προβλημάτων στις έξυπνες πόλεις καθώς θα χρειαστούν αν βγάλουν φωτογραφίες από ζημιές στον δρόμο και να τις υποβάλουν μαζί με την τοποθεσία τους. Το Capacitor το έκανε αυτό εφικτό χωρίς την προσθήκη συναρτήσεων κώδικα πάνω στην πλατφόρμα της συσκευής.

\subsection{Android και Android Studio}
Το Android μπορεί να θεωρηθεί παγκοσμίως το πιο ευρέως χρησιμοποιημένο λειτουργικό σύστημα για κινητά και τάμπλετ. Είναι χτισμένο πάνω σε μια αρχιτεκτονική με πολλά επίπεδα που περιλαμβάνει το kernel του Linux, ένα περιβάλλον εκτέλεσης, βιβλιοθήκες και framework εφαρμογών \parencite{goel2021android}. Αυτό το σχέδιο κάνει το Android ευέλικτο και επεκτάσιμο το οποίο ευθυγραμμίζεται ιδανικά με τα ζητούμενα των εφαρμογών για τις έξυπνες πόλεις που χρειάζονται αρκετές λειτουργίες. Από προγραμματιστική πλευρά, το ανοικτό μοντέλο του Android και διείσδυση του στην αγορά δικαιολογεί την επιλογή του ως ιδανική πλατφόρμα ανάπτυξης.
\\
\\
Ενώ το Ionic React και το Capacitor χειρίζονται τον κώδικα, το Android Studio είναι απαραίτητο για την παραγωγή του τελικού προγράμματος για Android. Σύμφωνα με την \textcite{thamizharasi2016android}, το Android Studio είναι ένα πλήρες περιβάλλον IDE όπου απλοποιεί την ανάπτυξη εφαρμογών με λειτουργίες όπως εξομοιωτές συσκευών και εργαλείο εντοπισμού σφαλμάτων. Η τυπική διαδικασία ροής του προγράμματος αυτού είναι η δημιουργία ενός έργου, το γράψιμο του κώδικα, η κατασκευή της εφαρμογής και η μεταφορά της στην συνδεμένη συσκευή ή προσομοίωση αυτής. Σε αυτή την εργασία, η χρήση αυτού του εργαλείου έγινε μόνο για το τελικό κομμάτι για την δημιουργία APK αρχείου από το έργο και την εγκατάσταση του στην δοκιμαστική συσκευή.  

\subsection{Συγκριτικές Επισκοπήσεις και Πρακτικές}
Αρκετές ακαδημαϊκές μελέτες έχουν συγκρίνει διάφορα frameworks πολλαπλών πλατφορμών όπως το Ionic, React Native και Flutter. Οι \textcite{majchrzak2017comprehensive} βρήκαν ότι το Ionic είναι το πιο κατάλληλο όταν είναι ζητούμενη η γρήγορη ανάπτυξη και η σταθερότητα στην διεπαφή χρήστη ενώ η React Native μπορεί να προσφέρει καλύτερη απόδοση σε εφαρμογές με μεγάλη αλληλεπίδραση. Σε μια άλλη μελέτη, αναφέρθηκε πως η εξάρτηση του Ionic πάνω στο WebView της μπορεί να επηρεάσει αρκετά την επίδοση περίπλοκων εφαρμογών αλλά στις πιο πολλές περιπτώσεις αυτό δεν αποτελεί σημαντικό περιορισμό \parencite{you2021comparative}. Οι \textcite{Sonar2023Hybrid} δίνουν έμφαση στο ότι το κύριο πλεονέκτημα του Ionic είναι η απλοϊκότητα του και η πλούσια βιβλιοθήκη που περιέχει όπου περιορίζει την ανάγκη για δημιουργία UI από την αρχή. Στην άλλη πλευρά, οι προγραμματιστές θα πρέπει να είναι προσεκτικοί ως προς την βελτίωση της απόδοσης όταν έχουν εφαρμόσει βαριά εφέ όπως κινήσεις και μεγάλα σετ δεδομένων. 
\\
\section{Σχετικά Έργα}
Όταν αναπτύσσεται μια κινητή εφαρμογή πάνω στις έξυπνες πόλεις, είναι σημαντική η κατανόηση παρόμοιων συστημάτων και των τεχνολογίων τους. Η ανάλυση άλλων έργων δίνει σημαντικές γνώσεις για οποιεσδήποτε προκλήσεις που πιθανά αντιμετοπίζουν και επιλογές στην σχεδίαση όπου μπορούν να εφαρμοστούν σε νέες υλοποιήσεις. Αυτό το τμήμα αξιολογεί σχετικές μελέτες περίπτωσης που αφορούν εφαρμογές για την αναφορά αστικών προβλημάτων.

\subsection{CitySolution}
Ένα αξιόλογο παράδειγμα είναι η περίπτωση του CitySolution, μια εφαρμογή για έξυπνες πόλεις φτιαγμένη από τους \textcite{shama2024citysolution}. Η πλατφόρμα αυτή σχεδιάστηκε για την βελτίωση της απόδοσης του Μπανγκλαντές ως προς την διαχείριση παραπόνων από τους δήμους, αντικαθιστώντας τις αργές και χειροκίνητες διαδικασίες με ψηφιακές. Το ίδιο χωρίζεται σε δύο διαφορετικά συστήματα, ένα για τους πολίτες και ένα για το προσωπικό των αρχών. Οι πολίτες μπορούν να αναφέρουν προβλήματα όπως κατεστραμμένους δρόμους καταχωρώντας απλά μια φωτογραφία με κείμενο και την τοποθεσία τους.
\\
\\
Μια κύρια λειτουργία του CitySolution ήταν η προσαρμογή μηχανικής μάθησης για την αυτόματη κατηγοριοποίηση των παραπόνων σε ήδη υπαρκτές κατηγορίες. Αυτό έγινε μέσω της χρήσης μοντέλου βαθιάς μάθησης εκπαιδευμένο από το Google Teachable Machine, κάτι που βοήθησε στην μείωση εργασίας από τους υπαλλήλους της πόλης και επιτάχυνε την όλη διαδικασία. Η εφαρμογή ακόμα υποστηρίζει άλλες γλώσσες για την ικανοποίηση των προτιμήσεων, κάνοντας την προσβάσιμη για περισσότερους χρήστες. Η διαχείριση των δεδομένων έγινε μέσω του Firebase, διασφαλίζοντας ενημερώσεις σε πραγματικό χρόνο και αλληλεπίδραση μεταξύ των πολιτών και των αρχών. 
\\
\\
Η ασφάλεια αντιμετωπίστηκε μέσω χαρακτηριστικών όπως ο έλεγχος των υπαλλήλων μέσω QR, η προσαρμογή παρακολούθησης των παραπόνων για τους πολίτες και συστημάτων που ειδοποιούν για ενημερώσεις της κατάσταστης τους. Αυτές οι προσθήκες συνολικά ενίσχυσαν την εμπιστοσύνη των πολιτών στο σύστημα. Ακόμα, η μελέτη αυτή ανέφερε πως πριν το CitySolution τα περισσότερα συστήματα παραπόνων ήταν ανεπαρκής και συχνά χρειαζόταν φυσικές επισκέψεις για την αναφορά ενός θέματος. Φέρνοντας αυτό το αυτοματοποιημένο σύστημα, ο νέος στόχος τους έγινε η ενεργή συμμετοχή των πολιτών και η βελτίωση ανταπόκρισης των δημόσιων υπηρεσιών.

\subsection{Εφαρμογή με χρήση Ionic}
Άλλη μία περίπτωση είναι αυτή από τους \textcite{dunka2017}, όπου μελετάνε την ανάπτυξη μια υβριδικής εφαρμογής για κινητά με λειτουργίες χαρτογράφησης. Η εκτέλεση της έγινε συνδυάζοντας το Ionic με το Leaflet για την προβολή γεωγραφικών δεδομένων πάνω σε μία διαδραστική διεπαφή χάρτη. Όπως εξήγησαν οι συγγραφείς, η εφαρμογή τους χτίστηκε ως ένα μονό έργο ικανό να τρέχει και σε Android και iOS μειώνοντας μελλοντική ταλαιπωρία διατήρησης διαφορετικών κώδικων. 
\\
\\
Η διαδικασία επικεντρώθηκε στην υλοποίηση του Leaflet για δυναμική απεικόνιση στον χάρτη ενεργοποιώντας χαρακτηριστικά όπως ζουμάρισμα και τοποθέτηση δεικτών. Οι λειτουργίες των συσκευών όπως η κάμερα και η γεωγραφική τοποθεσία έγιναν προσβάσιμες μέσω plugins. Οι ίδιοι ανέφεραν πως αυτή η προσέγγιση μείωση σημαντικά τον χρόνο ανάπτυξης σε σχέση με άλλες πιο παραδοσιακές μεθόδους ενώ διατηρούσαν την εμπειρία του χρήστη ικανοποιητική σε σχέση με τον στόχο της εφαρμογής.
\\
\\
Όσον αφόρα τους περιορισμούς, αναφέρθηκε πως οι υβριδικές εφαρμογές μπορεί πιθανά να αντιμετωπίσουν θέματα στην απόδοση όταν βρίσκονται σε υπερφόρτωση. Ωστόσο, για εφαρμογές όπου επικεντρώνονται πάνω σε οπτικοποίηση δεδομένων αυτά τα προβλήματα δεν έχουν σημαντική επίδραση. Τέλος, η μελέτη συμπεραίνει πως η χρήση framework όπως τo Ionic μπορεί να θεωρηθεί ιδανική για πλατφόρμες γρήγορης απόδοσης και γεωγραφικών λειτουργιών.

\subsection{Δίδαγμα από τις μελέτες}
Οι δύο μελέτες παρουσιάζουν ομοιότητες με το θέμα αυτής της εργασίας αλλά ταυτόχρονα μεταφέρουν σημαντικές διαφορές όπου επηρέασαν τις αποφάσεις κατά την σχεδίαση. Η πρώτη, κάνει επίδειξη της αποτελεσματικότητας που έχει η προσαρμογή χαρακτηριστικών όπως η αναφορά προβλημάτων μέσω φωτογραφιών, τοποθεσίας και αυτοματοποίησης σε κατηγορίες. Η τρέχον εργασία μπορεί να μην έχει εφαρμόσει μηχανική μάθηση για κατηγοριοποίηση αλλά παίρνει τις βασικές ιδέες ώστε να δώσει την δυνατότητα στους πολίτες να υποβάλλουν τις δικές τους αναλυτικές αναφορές. 
\\
\\
Σε σχέση με την δεύτερη, γίνεται αντιληπτή η σωστή επιλογή της χρήσης του Ionic σε αυτό το έργο. Η διατήρηση μια ενιαίας βάσης κώδικα είναι σημαντική και βοηθάει με τον στόχο δημιουργίας μιας προσιτής και αποδοτικής λύσης για τους δήμους. Ακόμα, η προσθήκη χαρτογράφησης μέσω Leaflet στην μελέτη τους ενισχύει την απόφαση υλοποίησης αυτής της τεχνολογίας για την οπτικοποίηση των αναφορών σε αλληλεπιδραστικό χάρτη.