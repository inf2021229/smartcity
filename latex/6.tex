\section{Συμπεράσματα}
\drop{H}{ολοκλήρωση} της εργασίας σηματοδοτεί την επιτυχή σχεδίαση και ανάπτυξη ενός ψηφιακού συστήματος στοχευμένη να βελτιώσει τον τρόπο όπου οι δήμοι διαχειρίζονται και ενεργούν πάνω σε ζητήματα υποδομών αναφερόμενα από τους πολίτες. Κατά την διάρκεια της διαδικασίας, το έργο έδειξε πως σύγχρονες τεχνολογίες ιστού και κινητών μπορούν να χρησιμοποιηθούν για την επίλυση καθημερινών προβλημάτων με ένα απλό και διαθέσιμο τρόπο. Συνδυάζοντας την εφαρμογή και την ιστοσελίδα, η πλατφόρμα προσφέρει ένα οργανωμένο τρόπο που συνδέει και τις δύο πλευρές του θέματος.
\\
\\
Μέσω της υλοποίησης του, επαληθεύεται πως μέχρι και μικρές κοινότητες μπορούν να επωφεληθούν από μίας ελαφριάς και ανοικτού κώδικα προσέγγισης για την ψηφιοποίηση των λειτουργιών τους χωρίς την ανάγκη πολύπλοκων συστημάτων που χρειάζονται μεγάλο αριθμό χρημάτων. Η επιλεγμένη ομάδα τεχνολογιών, η οποία αποτελείται από Ionic React, Node.js, Express και MongoDB, αποδείχθηκε πρακτική και υψηλά ικανή ώστε να αναπτυχθεί μία πλατφόρμα πολλαπλών διεπαφών. Η εφαρμογή για κινητά, δίνει με επιτυχία την δυνατότητα στους ανθρώπους να δημιουργήσουν, να περιγράψουν και να στείλουν αναφορές, ενώ η ιστοσελίδα επιτρέπει στους διαχειριστές να μπορούν να παρακολουθούν και να χειριστούν αυτές. Μαζί, δημιουργείται ένα πλήρες οικοσύστημα για αποδοτική αστική διαχείριση.
\\
\\
Από μία πιο γενική εικόνα, παρατηρείται και τονίζεται η σημαντικότητα του σχεδιασμού με προτεραιότητα στον χρήστη για λογισμικά δημόσιων υπηρεσιών. Η πλατφόρμα χτίστηκε με την απλοϊκότητα στο μυαλό, δίνοντας σημασία στην πρόσβαση και σαφήνεια της, ώστε να χρησιμοποιείται και από χρήστες χωρίς πλούσια τεχνολογική εμπειρία. Οι δοκιμές και τα αποτελέσματα των αξιολογήσεων σε σχέση με την εκτίμηση δείχνουν πως το σύστημα δουλεύει και αποδίδει ικανοποιητικά όσον αφορά την χρηστικότητα και την λειτουργία του. Άρα, τα αποτελέσματα αυτά επιβεβαιώνουν πως επιτεύχθηκαν οι αρχικοί στόχοι.
\\
\\
Συνολικά, η εργασία επιδεικνύει πως η τεχνολογία μπορεί να χρησιμοποιηθεί σωστά ως ένα εργαλείο. Η πλατφόρμα προσφέρει ένα πρακτικό παράδειγμα για το πως μία ψηφιακή μεταμόρφωση μπορεί να υλοποιηθεί και σε μικρότερα επίπεδα πόλεων. Τα αποτελέσματα και διδάγματα αυτής της διαδικασίας μπορούν να θεωρηθούν ως μία βάση για μελλοντικές βελτιώσεις και εξελίξεις συμβάλλοντας στο όραμα πιο έξυπνων και ενωμένων πόλεων.

\section{Μελλοντική Εργασία}
Παρόλο που η τρέχουσα κατάσταση της πλατφόρμας καλύβει με επιτυχία τους βασικούς της στόχους, υπάρχουν πολλά σημεία όπου μπορούν να βελτιωθούν και να επεκταθούν περαιτέρω στο μέλλον. Όσο η τεχνολογία εξελίσσεται, νέες ευκαιρίες εμφανίζονται όπου μπορούν κάνουν αυτό το σύστημα πιο ευφυής και προσαρμοσμένο στις ανάγκες των δήμων. Οι παρακάτω ιδέες αναφέρονται σαν πιθανές μελλοντικές κατευθύνσεις ανάπτυξης, στοχεύοντας στην ενίσχυση τόσο λειτουργικά όσο και χρηστικά.
\\
\\
Μία από τις πιο σημαντικές μελλοντικές βελτιώσεις μπορεί να θεωρηθεί η εισαγωγή των μοντέλων τεχνητής νοημοσύνης για την βοήθεια στην ανίχνευση και κατηγοριοποίηση των αναφορών. Για παράδειγμα, αναλύοντας τις φωτογραφίες που ανεβάζουν οι χρήστες, ένα σύστημα AI μπορεί να αναγνωρίσει αυτόματα το είδος του προβλήματος και να το ενσωματώσει στην σωστή κατηγορία. Αυτό μπορεί να σώσει χρόνο στους διαχειριστές αλλά και στους χρήστες κατά την υποβολή, κάνοντας την διαδικασία πιο αποδοτική. Επιπρόσθετα, μπορούν να εκπαιδευτούν αλγόριθμοι μηχανικής μάθησης ώστε να αναλύουν μοτίβα από τις αναφορές, βοηθώντας τους δήμους να προβλέψουν συχνά προβλήματα ή περιοχές που χρειάζονται πιθανή συντήρηση.

\newpage

Άλλο ένα σημαντικό βήμα για την καλύτερη προσβασιμότητα της εφαρμογής είναι η επέκταση σε άλλες πλατφόρμες και πιο συγκεκριμένα στις συσκευές iOS. Αυτή την στιγμή, η κινητή εφαρμογή αναπτύχθηκε και υλοποιήθηκε μόνο σε συσκευή Android. Με την επέκταση της υποστήριξης για τους χρήστες με iPhone, η εφαρμογή γίνεται προσβάσιμη σε μεγαλύτερο εύρος του πληθυσμού. Για να γίνει εφικτό αυτό απαιτείται ρύθμιση και ενημέρωση στον υπάρχοντα κώδικα της αφού το Ionic υποστηρίζει πολλαπλές πλατφόρμες χωρίς νέο κώδικα. Ακόμα, χρειάζονται περαιτέρω δοκιμές ώστε η τελική λειτουργία να είναι ίδια χωρίς κάποια προβλήματα όπως περιορισμοί της Apple.
\\
\\
Από την διοικητική πλευρά, οι μελλοντικές ενημερώσεις μπορούν να επικεντρωθούν στην προσθήκη περισσότερων εξελιγμένων εργαλείων. Αυτό θα μπορούσε να περιλαμβάνει λειτουργίες όπως πιο ανεπτυγμένο φιλτράρισμα και κατάταξη των προβλημάτων με βάση των δεδομένων τους. Ακόμα, μπορεί να υλοποιηθεί τρόπος εξαγωγής των δεδομένων σε υπολογιστικά φύλλα για σκοπούς τεκμηρίωσης και για την δημιουργία αυτόματων στατιστικών αναφορών. Άλλη μία βελτίωση μπορεί να είναι η ικανότητα ομαδοποίησης των αναφορών με βάση την γεωγραφική τοποθεσία καθώς και το είδος προβλήματος τους. Αυτό επιτρέπει στις πόλεις να οπτικοποιήσουν παρόμοια ή κοντινά θέματα ως συστάδες πάνω στο χάρτη, κάνοντας πιο εύκολη την αναγνώριση κρίσιμων περιοχών ώστε να σχεδιάσουν πιο στρατηγικά την προσέγγιση τους.
\\
\\
Επιπλέον, μπορούν να ενσωματωθούν ειδοποιήσεις σε πραγματικό χρόνο όπου ενδυναμώνει την επικοινωνιακή σχέση των χρηστών με τις αρχές και την εμπιστοσύνη τους στο σύστημα. Για παράδειγμα, οι πολίτες να δέχονται ενημερώσεις μέσω της εφαρμογής όταν αλλάζει η κατάσταση μίας αναφοράς τους και όταν αυτή ολοκληρωθεί, κάνοντας το σύστημα πιο ελκυστικό και αξιόπιστο. 
\\
\\
Συνολικά, αυτές οι βελτιώσεις έχουν στόχο να κάνουν την εφαρμογή πιο δυνατή και πλούσια σε χαρακτηριστικά καθώς και πιο προσαρμόσιμη σε μελλοντικές αστικές προκλήσεις. Η τωρινή έκδοση της χρησιμεύει ως μία σταθερή και ισχυρή βάση για μελλοντικές αναπτύξεις, προσφέροντας τρόπους για συνεχή καινοτομία.
